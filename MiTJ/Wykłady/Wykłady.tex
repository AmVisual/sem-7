\documentclass{article}
\usepackage{polski}
\title{Metody i Techniki Jądrowe - wykąłdy}
\author{Maciej Standerski}
\begin{document}
\maketitle

\section{Wykład (06.10.2023)}

\subsection{Wstęp}

Email: jan.pluta@pw.edu.pl

Strona: www.if.pw.edu.pl/\~pluta/pl/dyd/mtj

Wykłady: zdalnie

Wycieczki: 5-6 w piątki

Insytucje
CCB Kraków
SOLARIS

Zaliczenie: kolokwium (3 tematy do opracowania przez 1,5h, każdy z tematów punktwany jest na max 10 pkt, do zalieczenia kolokwium wymagane jest $51\%$ wszystkich punktów możliwych do zalieczenia. Kolokwium można zaliczyć w dwóch terminach. Kolokwium ustne (30 minut)).

\subsection{Wykład wstępny: Metody i techniki jądrowe w środowisku, przemyśle i medycynie}

Ewolucja wiedzy na temat struktury materii

Eksperyment Rutherforda

99,9\% masy atomu to masa jądra

Rozmiary jądra są 4 rzędy wielkości mniejsze niż rozmiary atomu

Cząstki fundamentalne i ich własności

Oddziaływania fundamentalne (model standardowy)

Potencjał w przypadku kwarków jest potencjałem rosnącym (musimy dostarczać więcej energii by rozdzielić parę kwarków, bo ich energia potencjalna rośnie, nie da się rozdzielić kwarków)

Aby uwolnić kwarki należy je podgrzać ($10^{12} K$!!!) lub ścisnąć ($\rho = 100 000 000 t/cm^3$)

Diagram fazowy materii hadronowej (przejście pomiędzy fazą hadronową a fazą kwarkowo-gluonową)

Ewolucja wszechświata
(Edwin Hubble - odkrycie, że wszechświat się rozszerza)

Plazma kwarkowo-gluonowa może stanowić centrum gwiazd neutronowych (prawdopodobnie!!!).

Obecność promieniowania na codzień:
- promieniowanie kosmiczne (istnieje wysokość, na której promieniowanie kosmiczne jest największe)

\end{document}