\documentclass{article}
\usepackage{polski}
\usepackage{hyperref}
\title{Metody i Techniki Jądrowe - wykąłdy}
\author{Maciej Standerski}
\begin{document}
\maketitle

\section{Wykład (06.10.2023)}

\subsection{Wstęp}

Email: jan.pluta@pw.edu.pl

Strona: \href{https://www.if.pw.edu.pl/~pluta/pl/dyd/mtj/}{www.if.pw.edu.pl/\~pluta/pl/dyd/mtj}

Wykłady: zdalnie

Wycieczki: 5-6 w piątki

Insytucje
CCB Kraków
SOLARIS

Zaliczenie: kolokwium (3 tematy do opracowania przez 1,5h, każdy z tematów punktwany jest na max 10 pkt, do zalieczenia kolokwium wymagane jest $51\%$ wszystkich punktów możliwych do zalieczenia. Kolokwium można zaliczyć w dwóch terminach. Kolokwium ustne (30 minut)).

\subsection{Wykład wstępny: Metody i techniki jądrowe w środowisku, przemyśle i medycynie}

Ewolucja wiedzy na temat struktury materii

Eksperyment Rutherforda

99,9\% masy atomu to masa jądra

Rozmiary jądra są 4 rzędy wielkości mniejsze niż rozmiary atomu

Cząstki fundamentalne i ich własności

Różnice między cząstkami elementarnymi a fundamentalnymi
Cząstki elementarne: proton, neutron, mezon, itd.
Cząstki fundamentalne: kwarki i leptony

Oddziaływania fundamentalne (model standardowy)

Potencjał w przypadku kwarków jest potencjałem rosnącym (musimy dostarczać więcej energii by rozdzielić parę kwarków, bo ich energia potencjalna rośnie, nie da się rozdzielić kwarków)

Aby uwolnić kwarki należy je podgrzać ($10^{12} K$!!!) lub ścisnąć ($\rho = 10^8 t/cm^3$)

Diagram fazowy materii hadronowej (przejście pomiędzy fazą hadronową a fazą kwarkowo-gluonową, $1 eV odpowiada 11000K$, na osi poziomej gęstość barionowa, na pionowej temperatura)

Ewolucja wszechświata
(Edwin Hubble - odkrycie, że wszechświat się rozszerza)

Plazma kwarkowo-gluonowa może stanowić rdzenie gwiazd neutronowych (prawdopodobnie!!!).

Plazmę kwarkowo-gluonową można uzyskać korzystając z metody zderzenia ciężkich jonów w akceleratorze.

Obecność promieniowania na codzień:
- promieniowanie kosmiczne (istnieje wysokość, na której promieniowanie kosmiczne jest największe)

13,8 miliardów lat temu narodził się wszechświat.

Główny skład dawki promieniotwórczej uzyskywanej na codzień:
\begin{itemize}
    \item radon - 40,6\%
    \item gamma - 13,8\%
    \item promieniowanie kosmiczne 8,5\%
    \item radionuklidy naturalne wewn. organizmu 8,3\%
    \item zastosowania medyczne, awaria czarnobyla, inne - 25,8\%
\end{itemize}

Radon - izotop promieniotwórczy będący wynikiem rozpadu radu. Znajduje się w powitrzu, dostaje się do niego z ziemi. Do budynków przenika przez szczeliny w ścianach.

Szkodliwość promieniowania alfa są 20 razy większa niż promieniowania gamma. Nie jest ono jednak bardzo przenikiliwe. Kartka papeiru jest w stanie je zatrzmać. Jednak jeśli źródła promieniowania alfa dostaną się do ogranizmu, mogą uszkodzić organy wewn.
Najbardziej przenikliwa jest wiązka neutronów.

Wykorzystanie promieniowania jonizującego:
\begin{itemize}
    \item defektoskopia przemysłowa
    \item techniki radiacyjne
    \item radiometryczna aparature przemysłowa
    \item analiza aktywacyjna
    \item znaczniki promieniotwórczej
    \item urządzenia do radiacyjnego utrwalania żywności
    \item energetyka jądrowa\slash termojądrowa
\end{itemize}

Instytucje jądrowe w okręgu warszawskim:
\begin{itemize}
    \item Narodowe Centrum Badań Jądrowych w Świerku
    \item Instytut Chemii i Techniki Jądrowej
    \item Państwowa Agencja Atomistyki
    \item Centralne Laboratorium Ochrony Radiologicznej
    \item Centrum Onkologii - Instytut im. Marii Skłodowskiej-Curie
    \item Instytut Fizyki Plazmy I Laserowej Mikrosyntezy
\end{itemize}

Układ okresowy posiada obecnie 118 pierwiastków. 

\section{Wykład (13.10.2023)}

\subsection{Budowa jądra atomowego}

Proton (kwarki uud)

Neutron (kwarki udd)

Model Thomsona atomu (Model ciasta z rodzynkami)

Eksperyment Ernesta Rutherforda (stworzył model planetarny atomu)

\subsection{Jak wyznaczyć rozmiary jądra atomowego}

Na podstawie ekesperymentu Rutherforda możemy wyznaczyć rozmiary jądra atomowego. Cząstka alfa wystrzelona bezpośrednio w kierunku jądra atomowego zostanie wyhamowana przez pole elektryczne jądra atomowego, po czym zostanie przez nie odepchnięta. Jeżeli nadamy cząstce alfa wystarczająco dużo energii kinetycznej, aby przezwyciężyła ona wpływ pola elektrycznego atomu, wówczas cząstka osiągnie odległość od centrum jądra atomowego równą promieniowi tego jądra. W momencie zetknięcia się cząstki alfa z jądrem atomowym modele rozporszenia cząstki przez jądro atomowe (eksperyment Rutherforda) przestanie obowiązywać, gdyż wtedy dojdzie wpływ sił jądrowych, których model nie uwzględnia. 

Jak można to wyznaczyć.

Energia kinetyczna cząstki alfa wynosi:
\begin{equation}
    E_k = \frac{mv^2}{2}
\end{equation}
Z drugiej strony potencjał kulombowski w odległości $r$ od jądra atomowego wynosi
\begin{equation}
    V = \frac{1}{4\pi \epsilon_0} \frac{q_{\alpha}Q_N}{r}
\end{equation}

Gdzie ładunek $q_{\alpha} = 2e$, natomiast ładunek jądra atomowego wynosi $Q_N = Z\cdot e$, gdzie $Z$ jest liczbą atomową badanego pierwiastka. Stąd równanie na potencjał można zapisać następująco:

\begin{equation}
    V = \frac{Z\cdot e^2}{2\pi \epsilon_0 r}
\end{equation}

W momencie zetknięcia się cząstki alfa z jądrem atomowym cała energia kinetyczna cząstki zrówna się z energią potencjalną:

\begin{equation}
    E_k = V \rightarrow R = \frac{Z \cdot e^2}{2\pi \epsilon_0 E_k}
\end{equation}

\subsection{Jak wyznaczyć masę jądra atomowego}
Spektometr masowy jest to urządzenie pozwalające na pomiar masy czątek oraz jąder atomowych poprzez przepuszczenie cząstki przez pole elektryczne i magnetyczne.
Na początku emitowana jest cząstka, która następnie rozpędzana jest w polu elektrycznym urządzenia. Cząstka nabywa wówczas energię kinetyczną, która wynosi:
\begin{equation}
    E_k = \frac{Mv^2}{2} = qU
\end{equation}
gdzie $U$ jest napięciem pola elektrycznego, w którym cząstka jest rozpędzana.
Następnie cząstka wpada w obszar pola magnetycznego. W polu magnetycznym na cząstkę działa siła Lorentza, która powoduje zakrzywienie toru ruchu cząstki. Promień, jaki zatacza cząstka w polu magnetyzcnym wynosi:
\begin{equation}
    r = \frac{Mv}{qB}
\end{equation}
Na podstawie tych dwóch równań możemy wyznaczyć masę cząstki:
\begin{equation}
    M = \frac{qr^2B^2}{2U}
\end{equation}

\subsection{Rozkład gęstości materii jądrowej}
Neutorny i protony są rozłożone w jądrze atomowym w sposób losowym, a nie jak by się mogło wydawać protony skoncentorwane na zewnątrz jądra z powodu działań sił kulombowskich, a neutrony w wewnątrz jądra. Wynika to stąd, że wewnątrz jądra atomowego większy wpływ mają siły jądrowe, a nie kulombowskie.

Wzór na promień jądra atomowego:
\begin{equation}
    R = R_0 \sqrt[3]{A}
\end{equation}
gdzie $R_0 \approx 1,3\cdot 10^{-15}~m = 1 \cdot 3~fm$ jest stałą która określa w przybliżeniu rozmiary protonu i neutronu, a $A$ - liczba masowa, czyli liczba nukleonów w jądrze atomowym.

\begin{equation}
    \rho (r) = \frac{\rho (0)}{1 + e^{r-R}a}
\end{equation}

\subsection{Energia wiązania jądra atomowego}

Wzór na energię wiązania (deficyt masy).
\begin{equation}
    E_w = \Delta Mc^2 = (Zm_p+Nm_n-M_j)c^2
\end{equation}
gdzie $Z$ - liczba atomowa (liczba protonów), $N$ - liczba neutronów, $m_p$ - masa spoczynkowa protonu, $m_n$ - masa spoczynkowa neutronu, $M_j$ - masa jądra atomowego.

Wykres średniej energii wiązania nukleonu w funkcji liczby masowej $A$. \textbf{Na kolokwium na pewno sie pojawi pytanie. Należy wiedzieć, co ten wykres opisuje i co z niego wynika.}
Maksymalna energia przypadająca na nukleon przypada dla żelaza Fe-58 i wynosi ok. $9~MeV$.

\subsection{Model kroplowy}
Model ten stanowi analogię kropli cieczy, która skubpia zbiór cząstek i jeśli nie jest poddana działaniu czynników zewn., zachowuje trwałość i kształ zbliżon do kulistego. W przypadku jądra rozpatruje się następujące czynniki określające enerigię wiązania nukleonów:

\begin{equation}
    E_w = E_{obj} + E_{pow} + E_{Coul} + E_{sym} + E_{par}
\end{equation}
gdzie $E_{obj} = a_0 A$ - czynnik objętościowy, $E_{pow} = -a_p A^{2/3}$ - czynnik powierzchniowy, $E_{Coul} = -a_c Z^2 A^{-1/3}$ - czynnik kulombowski, $E_{sym} = -a_s (A-2Z)^2/A$ - czynnik symetryczny, $E_{par} = \delta A^{-3/4}$ - czynnik wynikający z tendencji do łączenia się nukleonów w pary.

\subsection{Siły jądrowe}
Oddziaływanie silne działające pomiędzy kwarkami.
Wykres sił działających na nukleony w jądrze atomowym. \textbf{(na kolokwium)}

Własności sił jądrowych:
\begin{itemize}
    \item siły jądrowe nie zależą od ładunku elektrycznego
    \item są krótkozasięgowe, rzędu $10^{-15}~m$.
    \item charakteryzuje je własność wysycania. Każdy nukleon oddziałuje tylko z najbliższymi sąsiadami.
    \item zależne od wzajemnej orientacji spinów nukleonów, nie tylko od odległości między nukleonami
\end{itemize}

\subsection{Model gazu Fermiego}
Model oparty na tym, że nukleony nie są traktowane jako związane, ale swobodne. Swoboda ta jest jednak przez barierę studni przyciągającego potencjału jądra. Obowiązuje w nim zakaz Pauliego, zakazującego dowolnym dwum nukleonom obsadzenia tego samego stanu energetycznego.

Co to są liczby kwantowe -  liczby opisujące dyskretne wielkości fizyczne[1], np. poziomy energetyczne cząstek[2], atomów, jąder atomowych, cząsteczek gazów, elektronów swobodnych czy w sieci krystalicznej itd.
\href{https://pl.wikipedia.org/wiki/Liczby_kwantowe}{Wikipedia - Liczby kwantowe.}
Cztery główne liczby opisujace stan kwantowy:
\begin{itemize}
    \item głowna - odpowiada za energię
    \item poboczna - odpowiada za moment orbitalny
    \item magnetyczna - odpowiada za wartość rzutu momentu orbitalnego na wyróżnioną oś
    \item spinowa - odpowiada za wartość spinu
\end{itemize}

Chromodynamika kwantowa (teoria).

Jeżeli np. proton osiągnie energię potencjalną większą niż studnia potencjału, to wówczas może wydostać się z tej studni. Jej energia potencjalna zamienia się na energię kinetyczną. Oznacza to również, że nie możemy zauważyć (w więkoszści przypadków, wyjątek stanowią przypadki, gdy w grę wchodzi zjawisko tunelowania kwantowego) emisji cząstek z jąder atomowych o energiach niższych niż górna granica potencjału studni, bowiem dla niższych energii cząstki nie są w stanie wydostać się ze studni potencjału jądra atomowego.

Dodatkowo, aby proton mógł doprowadzić do rozszczepienia jądra atomowego, musi on mieć na tyle dużą energię, aby pokonać barierę studni potencjału. Natomiast w przypadku neutronów efekt ten zachodzi przy niskich energiach. Samo wprowadzenie neutronu do jądra atomowego powoduje jego destabilizację i może prowadzić do rozszczepienia jądra i wydzielenia przy tym ogromnej ilości energii.

\subsection{Model powłokowy}
Opisuje zachowanie "jąder magicznych" (2,8,20,28,50,82,126,...).
Model opisuje, iż nukleony znajdują się na powłokach.

\section{Wykład (20.10.2023) - źródła promieniotwórcze}

Aktualnie 118 peirwiastków zarejestrowanych.

\subsection{Przemiany promieniotwórcze}

Przekształenie jądra w inne z równoczesną emisją cząstki, $\alpha$, $\beta$ lub $\gamma$. Przemiany promieniotwórcze należą do klasy procesów stochastycznych.  (Przemiany chemiczne nie mają wpływu na zmiany jądra atomowego).

Przemiana alfa:
\begin{itemize}
    \item emitowane z atomów ciężkich (liczba masowa rzędu 200)
    \item jądro atomu helu
    \item wzór na przemianę alfa
    \item im większa jest energia emitowanej cząstki tym mniejsza jest szerokość bariery potencjału i tym bardziej jest prawdopodobny rozpad
    \item energia uwalniana w rozpadzie określona jest przez różnicę mas jądra macierzystego i produktów rozpadu $Q_a = (M(X)-[M(Y)+M(a)])\cdot c^2$, $Q_a = ([E_w(Y)+E_w(a)] - E_w(Y))\cdot c^2$
    \item w rezultacie przemiany alfa z jądra wyemitowana jest cząstka alfa
    \item energia rozpadu $Q_a = E_k(\alpha)\left[ 1 + \frac{m_{\alpha}}{m_{Y}}\ \right]$
    \item energia emitowanych cząstek alfa dla danej przemiany jest dobrze określona, co jest konsekwencją faktu, że rozpad jest dwuciałowy
\end{itemize}

Rozpad beta:
\begin{itemize}
    \item beta "+" i beta "-"
    \item w trakcie przemiany beta "-" następuje zmiana neutornu w proton i emisja elektronu oraz antyneutrina elektronowego
    \item Rozpad beta jest rozpadem trzyciałowym
    \item w trakcie rozpadu emitowane jest 
    \item przemiana beta-minus może zachodzić, jeśli masa atomu pocz jest mniejsza od masy atomu końcowego
    \item w trakcie rozpadu beta-plus następuje zamiana protonu w neutron i emitowany jest pozyton i neutrino elektornowe
    \item energie pocz pozytonów są większe niż
    \item wychwyt elektronów - proces zbliżon do procesu beta-plus
    \item z orbity atomowej przez jądro atomowe 
    \item warunek wychwytu $M_A(Z,A) > M_A (Z-1, A)$
    \item następuje wówczas również emisja fotonu rentgenowskiego oraz emisja elektronu z bardziej zewnętrznej powłoki atomu, zwanym elektornem Augene'a.
\end{itemize}

Ścieżka stabilności - stanowią ją wszystkie stabilne jądra atomowe.
Dla jąder lekkich prosta jest zbliżona do prostej $N=Z$. Im większa liczba atomowa, tym bardziej śceiżka odbiega od tej prostej. Aby jądro było stabilne, siły jądrowe możą równoważyć siły oddziaływania kulomboweskiego. Dla cięższych jąder odpowiednia liczba neutronów zapewnia stabilność.

Przemiana gamma:
\begin{itemize}
    \item towarzyszy rozpadom alfa lub beta. Cząstka gamma emitowana jest w trakcie przemiany wzbudzonego jądra atomowego do stanu podstawowego.
    \item konwersja wewnętrzna - proces, w którym miejsce emisji fotonu następuje wyemitowanie elektronu z jednej z powłok atomowych. $X_Z^A + e_{at}^- \rightarrow X_Z^A + e^-$. Współczynnik wewnętrznej konwersji $\alpha = \frac{p_e}{p_{\gamma}} = \frac{N_e}{N_{\gamma}}$
    \item stan wzbudzony nazywamy stanem izomerycznym, przejście izomeryczne (zachodzi z pewnym opóźnieniem)
    \item widmo elektronów emitowanych w wyniku rozpadu jąder atomowych patrz wykład.
\end{itemize}

Rozpad jądrowy z emisją protonu - mało prwadoposobne

Rozpad jądrowy z emisją neutronu

Rozszczepienie spontaniczne (przykładowo rozpad kalifornu-252)

\subsection{Schematy promieniotwórcze}

Nuklid początkowy, typ ropadu, nuklid końcowy

\subsection{Prawo rozapdu promieniotwórczego}

24.11 - spotkanie obecność obowiązkowa

Rozpad promieniotwórzy nie jest wynikeim wpływu czynników zewnętrznych na jądro atomowe i nie zależy od wcześniejszych jego losów. Rozpad ma charakter stochastyczny.

\begin{equation}
    N(t) = N_0 \cdot \exp{-\lambda t}
\end{equation}
gdzie $\lambda$ jest współczynnikiem zwanym stałą rozpadu, $N_0$ jest początkową liczbą jąder.

Przy rozpadzie przez bardzo krótki przedział czasu, rozumiemy taki czas, dla którego liczba jąder po rozpadzie jest zbliżona do liczby jąder przed rozpadem. Np. 5 rozpadów na 10 jąder nie będzie krótkim czasem, ale już 5 rozpadów na 1000000 jąder będzie krótkim czasem. 

Liczba jąder, które uległy rozpadowi:
\begin{equation}
    N_0 - N(t) = N_0\cdot (1-e^{-\lambda t})
\end{equation}

Średni czas życia: $\tau = 1/\lambda$

Czas połowicznego zaniku\slash rozpadu - taki czas w którym połowa jąder nie ulega rozpadowi

\begin{equation}
    T = \frac{\log{2}}{\lambda} = \log{2} \cdot \tau
\end{equation}

Aktywność źródła - liczba rozpadów zachodzących w jednostce czasu.

\begin{equation}
    A = \lambda \cdot N_0 \cdot e^{-\lambda t} = A_0 \cdot e^{-\lambda t} = A_0 \cdot e^{-\log{2} \cdot t / T}
\end{equation}

\subsection{Sukcesywne rozpady promieniotwórcze}

Występuje, gdy po rozpadzie jednych jąder promieniotwórczych powstają inne jądra promieniotwórczegorcze. Jaka jest aktywność takiego układu, gdzie jedne jądra powstają, a inne ulegają rozpadowi?

Szereg promieniotwórczy

Przykład dla przypadku jąder promieniotwórczych typu 1 i 2, gdzie 2 powstaje z rozpadu 1, a z rozpadu 2 powstaje jądro stabilne.

\subsection{Naturalne źródła promieniotwórcze}

Szeregi promieniotwórcze

Pierwiaski naturalne muszą mieć czas połowicznego rozadu porównywalne z czasem istnienia Ziemi (rzędu miliardów lat).

Istnieją cztery szeregi promieniotwórcze:
torowy, naptunowy, uranowo-radowy, uranowo-aktynowy.

\subsection{Sztuczna promieniotwórczość}

Źródła neutronów, to na ogół układy izotopowe, w których wykorzystuje się źródło cząstek naładowanych. Cząstki te wywołują następnie reakcje jądrowe w których emitowany jest neutron.

Często stosowane jest źródło radowo-berylowe.

Reakcje typu $(d,n)$ - tzw. generatory neutronów

$_1^3T + _1^2D \rightarrow _0^1n + _2^4He, Q=+17,586~MeV$

Reakcja będąca podstawą energetyki termojadrowej

Źródła rozszczepieniowe

Produkcja neutronów w reaktorach termojądrowych

\subsection{Wiązki promieniowania jonizującego}

Wiązka promieniowania jonizującego- strumień cząstek naładowanych, jonizujących emitowanych w określnym kierunku.

Źródła promieniotwórcze tworzą cząstki o określonych energiach. W zastosowaniach medycznych, przemysłowych potrzebne są cząstkich o wyższcych energiach.

Urządzenia służące do przyspieszania cząstek naładowanych nazywamy akceleratorami.

Częstość cyklotronowa: $\omega = \frac{2\pi}{T} = \frac{qB}{m}$ (Alvares) - częstotliwość, w jakiej naładowana cząstka o ładunku $q$ i masie $m$ zatacza okrąg w polu magnetycznym o indukcji $B$. Ważne - częstość ta nie zależy od prędkości cząstki.

\subsection{Spektrometry magnetyczne i akceleratory}

Spektrometr masowy (prof. Danysz).

Spektrometr magnetyczny

Akceleratory: liniowe i kołowe

Generator Greinachera lub Cockrofta-Waltona

Generator Van de Graffa (domianą detektora typu Van de Graffa jest tzw. Pelletron)

Akceleratory typu tandem

\end{document}