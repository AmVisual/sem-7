\documentclass{article}
\usepackage{polski}

\usepackage{hyperref}
\usepackage[margin=2.5cm]{geometry}

\setlength{\parindent}{0pt}

\title{Wstęp do Fizyki Medycznej - wykłady}
\author{Maciej Standerski}
\begin{document}
\maketitle

\section{Wykład (09.10.2023)}

\subsection{Wstęp}

Email: piotr.tulik@pw.edu.pl

Zaliczenie kolokwium (test: forma zamknięta i otwarta)

Maksymalna liczba do zdobycia na kolokwium: 30

Zaliczenie od 51\% punktów z kolokwium.

Literatura:

"Planowanie leczenia i duzymetria w radioterapii", J.Malicki, K. Ślisarek

"Biocybernetyka i inżynieria biomedyczna 2000" i "Fizyka medyczna", Maciej Nałęcz

"Fizyczne metody diagnostyki medycznej i terapii", A Hrynkiewicz, E. Rokita

"Dozymetria promieniowania jonizującego w radioterapii" i "Podstawy fizyki promieniowania jonizującego..." W. Łobodziec

\subsection{Czym jest fizyka medyczna}

Jakie dziedziny obejmuje fizyka medyczna:
\begin{itemize}
    \item diagnostyka
    \item radioterapia (teleradioterapia i brachyterapia)
    \item medycyna nuklearna (terapia izotopowa, SPECT, itd.)
    \item inżynieria biomedyczna
    \item fizykoterapia (laseroterapia, elektroterapia, galwanizacja, itd.)
    \item kontrola jakości
    \item steryzlizacja radiacyjna
    \item ochrona radiologiczna
    \item zjawiska elektromagnetyczne w diagnostyce i tomgrafia MR
\end{itemize}

Tomografia (PET, SPECT, tomografia rezonansu magnetycznego)

\subsection{Radioterapia}

Radioterapia - miejscowa metoda leczenia nowotworów, wykorzystujące energię promieniowania jonizującego. Stosowana w onkologii do leczenia chorób nowotworowych oraz łagodzenia bólu i innych dolegliwości związanych z rozsianym procesem nowotworowym, np. w przrzutach nowotworowych do kości lub guzach powodujących niedrożność oskrzela.

Promieniowanie można podzielić ze względu na:

\textbf{Oddziaływanie z materią}
\begin{itemize}
    \item Promieniowanie pośrednio jonizujące elektromagnetyczne: X i gamma (rozróżniamy na podstawie źródła pochodzenia: promieniowanie $\gamma$ jest promieniowaniem rentgenowskim powstającym podczas przemian jądrowych, natomiast promieniowanie X jest promieniowaniem emitowanym przez cząstkę poruszającą się ruchem przyspieszonum)
    \item Promieniowanie cząstkowe (bezpośrednio jonizujące, pośrednio jonizujące (neutrony))
\end{itemize}

\textbf{Energię}
\begin{itemize}
    \item Radioterapia konwencjonalna (60 d0 400 keV)
    \item Radioterapia megawoltowa (1,25 do 25 MeV)
    \item Elektrony (6 do 22 MeV)
    \item Wiązki hadronowe (60 do 230 MeV)
\end{itemize}

Zalety megawoltowego promieniowania X
\begin{itemize}
    \item większa przenikliwość
    \item mniejsza zdolność pochłaniania przez tkankę kostną
    \item lepsza tolerancja leczenia
\end{itemize}

Brachyterapia - umieszczenie źródła promieniowania w jamach ciała, bezpośrednio w guzie albo w jego otoczeniu, najczęściej wykorzystywanym radiofarmaceutykiem jest izotop irydu 192 (źródło zamknięte lub źróło otwarte). Rozróżnia się brachyterapię:
\begin{itemize}
    \item wewnątrzkomórkową - umieszczenie źródła w guzie
    \item wewnątrzjamową - umieszczenie źródła w bezpośrednim sąsiedztwie guza przy użyciu naturalnych otworów w ciele
    \item powierzchniową - umieszczenie źródła na skórze w celu leczenia zmian powierzchniowych
    \item śródnaczyniową - źródła umieszczone w naczyniach krwionośnych
    \item śródoperacyjną
\end{itemize}

Teleradioterapia - technika leczenia za pomocą promieniowania jonizującego (radioterapia), w metodzie tej źródło promieniowania umieszczone jest w pewnej odległości od tkanek. Polega na napromienianiu wiązkami zewnętrznymi określonej objętości tkanek, obejmującej guz nowotworowy z adekwatnym marginesem tkanek oraz, w razie potrzeby, regionalne węzły chłonne. Jednym z rodzajów terapii jest terapia hadronowa, w której wyróżnia się:
\begin{itemize}
    \item terapię cząstkami naładowanymi takimi jak protony, jony, ujemne mezony pi
    \item terapię cząstkami neytralnymi, czyli neutronami, w której wyróżnia się natomiast FNT (terapia szybkimi neutronami) oraz BNCT (terapia borowo-neutronowa)
\end{itemize}

Terapia izotopowa - podanie izotopu promieniotwórczego (najczęściej jod, w diagnostyce glukoza jako nośnik)

Skutki promieniowania: deterministyczne (gdy zostanie przekroczony próg) i stochastyczne (występują zawsze)

Frakcjonowanie - podział na mniejsze dawki

\subsection{Historia promieniotwórczości}

Bomba radowa 

Kliniczny akcelerator van de Graaff'a

Bomba kobaltowa - urządzenie do teleradioterapii lub napromieniowywania przedmiotów promieniami gamma ($\gamma$) o energiach 1,17 i 1,33 MeV, emitowanymi przez izotop kobaltu 60Co o aktywności rzędu 1013-1014 Bq. Ze względu na dużą przenikliwość promieniowania gamma aktywny kobalt jest otoczony grubą osłoną z ołowiu, w której znajdują się kanały wyprowadzające na zewnątrz wiązkę promieniowania. Bomba kobaltowa może też być wyposażona w mechanizm umożliwiający zdalną manipulację próbkami bez narażania operatora na promieniowanie. Bomba kobaltowa jest stosowana w lecznictwie do zwalczania chorób nowotworowych, w defektoskopii, do sterylizacji żywności oraz w chemii radiacyjnej, do badań procesów fizykochemicznych zachodzących podczas napromieniowywania wysokoenergetycznymi kwantami gamma prostych i złożonych układów chemicznych. Nazwą tą określana jest także broń jądrowa z płaszczem kobaltowym.

Insytuty medycyny nuklearnej w Polsce:

1932 - otwarcie Instytutu Radowego w Warszawie

Instytut Onkologii im Marii Skłodowskiej Curie

Narodowy Insytut Onkologii im. Marii Skłodowskiej Curie Instytut Badawczy

Cyklotronowe Centrum Bronowice IFJ PAN (cyklotron AIC-144, Cyklotron Proteus C-235)

Obrazowanie planarne: radiografia ogólna, fluoroskopia, mammografia, stomatologia, densytometria

Scyntygrafia

\subsection{Lampa rentgenowska}

Anoda posiada wolframową powierchnię (posiada wysokoą liczbę atomową $Z = 74$ oraz wysokoą temperaturę topnienia i niski wskaźnik parowania) wtopioną w miedzianą tarczę. Powierzchnia znajdująca się na anodzie może być również wykonana z renu, a w przypadku lamp stosowanych w mammografii może być również wykonana z molibdenu ($Z=42$) ze względu na odpowiednią energię powstającą w wyniku zderzeń elektronów z tarczą anody.

Lampy rentgenowskie ze stałą oraz z wirującą anodą

$N = cT^2e^{-dT}$

Całkowita energia promieniowania rentgenowskiego

$W = kZE_0^2$

Lampy mogą być wolnoobrotowe lub szybkoobrotowe.

Na żywotność lampy zasadniczy wpływ mają łożyska (przy używaniu lampy należy oszczędzać łożyska)

Anoda stacjonarna (powierzchnia rzędu $4~mm^2$)

Anoda wirująca ($1835~mm^2$, wynikiem stosowanie tego rodzaju lampy jest większe liczba fotonów emitowanych z powierchni i dzięki tamu samym krótszy czasu ekspozycji. Dzięki temu, że lampa nie jest bombardowana przez elektrony tylko w okolicach jednego punktu, tylko na całej powierzchni anody, znacznie wolniej się ona nagrzewa.)

Ognisko elektryczne, rzeczywiste i optyczne (pozorne) (im większe ognisko, tym gorsza rozdzielczość obrazu)

Kąt nachylenia anody

Szklana obudowa - utrzymanie próżni $10^-6 mmHg$, odizolowanie elektrod, zespolenie katody i anody.

Kołpak ochronny - chroni przed wydostaniem się promieniowania w niepożądanym kierunku. (olej transformatorowy)

Moc lampy 

Co wpływa na uszkodzenie lampy:
\begin{itemize}
    \item Zbyt długi czas ekspozycji
    \item Zbyt krótki czas pomiędzy ekspozycjami
    \item Odparowanie katody
\end{itemize}

Typy aparatów rentgenowskich:
\begin{itemize}
    \item Aparat typu głowicowego (zasilacz wysokiego napięcia + lampa RTG w kołpaku)
    \item Aparat typu kołpakowego (zasilacz stanowi oddzielne urządzenie)
\end{itemize}

Zasilacze wykorzystywane przy aparatach rentgenowskich:
\begin{itemize}
    \item jednopulsowe
    \item dwupulsowe
    \item sześciopulsowe
    \item dwunastopulsowe
\end{itemize}

np. Zasilacze impulsowe WCZ

Rodzaje ograniczników:
\begin{itemize}
    \item stałe
    \item nastawne
    \item głębinowe
    \item irysowe
    \item uciskowe
\end{itemize}

Stopień nieostrości geometrycznej $n = \frac{s*p}{f-p}$

Efekt półcienia - powstaje, gdy duże ognisko znajduje się w małej odległości od obrazowanego obiektu. Można go zminimalizować oddalając ognisko.

Czynniki wpływające na jakość zdjęcia:
\begin{itemize}
    \item dobór warunków ekspozycji
    \item wielkość ogniska lampy rentgenowskiej (im mniejsze ognisko tym lepsza zdolność rozdzielcza)
    \item wartość i rodzaj zastosowanej filtracji całkowitej (dodatkowa filtracja powoduje zmniejszenie dawki, którą pacjent pochłonie (można ograniczyć ilość miękkiego promieniowania, która nie dodaje nic do diagnozy), jednak powoduje wzrost ilości rozproszonego promieniowania, co przekłada się na niższy kontrast zdjęcia)
    \item stosowanie kratki przeciwrozproszeniowej - zapobiega przedostawaniu się promieniowania rozproszonego
    \item odległość ognisko lampy-badany obiekt-rejestrator obrazu - wpływa na ostrość obrazu, małe ognisko oraz niewielka odległość badanego obiektu od rejestratora daje lepszą ostrość
\end{itemize}

Rodzaje filtrów:
\begin{itemize}
    \item Filtry rentgenowskie - zmiana widma promieniowania przez zastosowanie ośrodka pochłaniającego
    \item Filtr własny - bańska szklana, okno kołpaka, olej transformatorowy
    \item Filtry dodatkowe - mocowany na zewnątrz kołpaka, może być absorpcyjny, charakterystyczny lub kompensacyjny
    \item Filtracja całkowita - suma filtracji własnej i dodatkowej 
\end{itemize}

Kratki przeciwrozproszeniowe (współczynnik wypełnienia, liczba $linii/cm$, gęstość powierzchniowa ołowiu w kratce $g/cm^2$, efektywność kratki, absorpcja, jakość kratki)

3 główne parametry obrazowania:
\begin{itemize}
    \item napięcie lampy - różnica potencjałów przyłożonych do anody i katody lampy rentgenowskiej. Zwykle napięcie lampy rentgenowskiej jest wyrażone przez wartość szczytową w kilowoltach ($kV$). Im wyższa wartość napięcia, tym krótsza fala promieniowania, wyższa energia i przenikliwość, a co za tym idzie wyższe „zaczernienie” obrazu;
    \item prąd lampy - prąd elektryczny wiązki elektronów padających na tarczę lampy rentgenowskiej. Zwykle prąd lampy rentgenowskiej jest wyrażony wartością średnią w miliamperach (mA). Prąd katody determinuje jej temperaturę, im wyższa temperatura, tym większa ilość emitowanych elektronów, a co z tym związane – większa ilość kwantów promieniowania
    \item czas ekspozycji - czas trwania napromieniania, zdefiniowany zależnie od określonej metody, zwykle czas, w którym moc wielkości promieniowania przekracza określony poziom. Im dłuższy czas ekspozycji, tym większa dawka, a więc i „zaczernienie” obrazu.
\end{itemize}

Tryby pracy aparatu RTG:
\begin{itemize}
    \item technika dwupunktowa (parametrami są napięcie lampy [kV] oraz obciążenie prądowo-czasowe [mAs])
    \item technika trzypunktowa (3 główne parametry obrazowania)
\end{itemize}

Parametry obrazu rentgenowskiego:
\begin{itemize}
    \item rozdzielczość obrazu
    \item kontrast obrazu
    \item ostrość obrazu
\end{itemize}

\href{https://brain.fuw.edu.pl/edu/index.php/Obrazowanie:Obrazowanie_Medyczne/Metody_obrazowania_medycznego_wykorzystuj%C4%85ce_promieniowanie_rentgenowskie}{Link do strony Brain Wiki: "Obrazowanie Medyczne$/$Metody obrazowania medycznego wykorzystujące promieniowanie rentgenowskie}

\section{Wykład (16.10.2023)}

Układ AEC

Układ IBS - fluoroskopia

HU - jednostka obciążenia cieplnego anody (ma wymiar energii)

Dla generatora jednopulsowego
$\mathrm{HU} = 1 kV \cdot mA \cdot s$

Sześciopulsowego
$\mathrm{HU} = 1,35 kV \cdot mA \cdot s$

Dwunastopulsowego
$\mathrm{HU} = 1,41 kV \cdot mA \cdot s$

\subsection{Mammografia}

W niektórych przypadkach istnieje szczególna konieczność zmniejszenia rozproszeń.

Badanie radiograficzne tkanki miękkiej

Wartości gęstości i efektywnego Z dla wybranych tkanek ludzkich

Tkanka gruczołowa, tłuszczowa i włóknista

Linear attenutaion coefficient (problem odróżnienia tkanki włóknistej i gruczołowej - przy niskich energiach promieniowania różnica jest możliwa do zaobserwowania (ok. 20 keV), efekt fotoelektryczny)

X oddziałuje z tkanką w skutek rozproszenia komptonowskiego i efektu fotoelektrycznego

Konwencjonalne lampy RTG emitują promieniowanie o energiach 70-100 keV.

Energia w zakresie 18-23 keV w zależności od grubości i składu piersi.

Wykres optymalnych energii dla mammografii.

Wykorzystanie promieniowania charakterystycznego (widmo z anody wolframowej filtrowane molbdenem lub Rh)

Zestawienia anoda/filtracja
\begin{itemize}
    \item Mo-Mo
    \item Mo-Rh
    \item Rh-Rh
    \item Wolfram-Rh 
\end{itemize}

Molibden Z = 42, Wolfram Z = 74, przez co w anodzie molibdenowej dominuje promieniowanie charakterystyczne

Rh (Z = 45)

Zakres napięć anody: 24-28 keV. Jeśli napięcie lampy jest zbyt niskie, wartość mAs może rosnąć do nieakceptowalnych wartości, zwiększając niebezpiecznie dawkę promieniowania.

Mammograf - budowa

Płytka kompresująca (zmniejsza się grubość sutka, a więc zmniejsza się rozporoszenie promieniowania i dzięki temu rośnie rozdzielczość obrazu)

Wiązka promieniowania musi być odpowiednio ułożona (kolimacja wiązki) - system jest tak zubodowany, że pionowa wiązka przechodzi równolegle do ciała pacjentki.

Rozmiar ogniska jest bardzo istotnym parametrem lampy mammograficznej.

System DIcom

Rozdzielczość paru linii\slash cm.

Mammografy mają dwie wielkości ogniska: 0.3 i 0.1 mm

ACR accredition phantom

Tomosynteza

Aparaty stomatologiczne
\begin{itemize}
    \item wewnątrzustne
    \item zewnątrzustne (pantomografia (przygotowania do założenia aparatu na zęby), cefalografia)
    \item aparaty 3D
\end{itemize}

Aparaty z ramieniem C (radiologia zabiegowa, angiografia)

Aparaty densytometryczne

Aparaty przyłóżkowe

Tomografia klasyczna

\subsection{Podstawowe cechy nowotworów}

KRN - krajowy rejestr nowotworów

Nowotwór - grupa chorób, w których komórki organizmu dzielą się w niekontrolowany sposób, a nowo powstałe komórki nowotworowe nie różnicują się w typowe tkanki.

Nowotwór - czyli niekontrolowana proliferacja komórek. Wszystkie komórki nowotworowe są klonami pojedynczej komórki.

Żeby powstał nowotwór złośliwy wymaga to zgromadzenia kilku mutacji w komórce (6-8).

1. Mutacje w genach kontrolujązych mitozę
2. W genach regulujących proces apoptozy (zaprogramowanej śmierci komórki)

4. Stymulujące angiogenezę - tworzenie się nowych naczyń krwionośnych niezbędneych do rozwoju guza nowotworowego
5. Stymulujące powstanie przerzutów - roznosić drogą układu ochronnego i krwionośnego

Cechy charakterystyczne nowotworów:
\begin{itemize}
    \item Szybki wzrost
    \item Naciekanie i niszczenie okolicznych tkanek
    \item Zdolność do przerzutów do węzłów chłonnych i odległych narządów
    \item Zaburzenia apoptozy
    \item Anaplazja
    \item Wznowy miejscowe
    \item brak torebki guza
    \item odrastanie w miejscu występowania po niedokładnym usunięciu pierwszej zmiany
    \item duża zdolność tworzenia nowych kompensacyjny
    \item duże zróżnicowanie wyglądu komórek nowotworowych
\end{itemize}

Cykl komórkowy - fazy
\begin{itemize}
    \item Mitoza (6-8)
    \item wzrost G1 (1-8)
    \item synteza DNA S - replikacja (6-8 h)
    \item wzrost G2 (2-4)
\end{itemize}

Cykl komórkowy w komórkach nowotworowych jest krótszy niż w zdrowych.

Proliferacja - zdolność rozmnażania się komórek; jedna z cech organizmów żywych

Kontrola cyklu komórkowego

Częstość wchodzenia w fazy S i M cyklu jest różna w zależności od rodzaju komórki.

Nadmierna apoptoza jest hamowana przez "geny przeżycia".

Onkogeneza (przez 5 lat pacjent uznawany jest jako pacjent onkologiczny)

Radiowrażliwość tkanki nowotworowej i prawidłowej - komórki wykazują różną wrażliwość na promieniowanie jonizujące. Wykresy prawd-stwa kontroli nowotworu

Stosując radioterapie dązy się do jak największego TCP i jak najmniejszego NTCP (zazwyczaj 0,05).

Radiochirurgia

Terapia radioizotopowa

Terapia radykalna - prowadzona z zamiarem wyleczenia choroby nowotworowej, i paliatywna - poprawa jakości życia w okresie kiedy choroby nie można zatrzymać

Nowotwór to nieprawidłowa tkanka powstająca z jednej "chorej" komórki organizmu. Nowotwory mogą być łagodne i złośliwe.

Rozwój nowotworu:
\begin{itemize}
    \item Zmiana przednowotworowa - wiąże się z rysykiem rozwoju nowotworu złośliwego
    \item Stan przednowotworowy - choroba związana ze zwiększonym ryzykiem wystąpienia nowotworu złośliwego
\end{itemize}

Nowotwory złośliwe:
\begin{itemize}
    \item pochodzenia nabłonkowego (raki)
    \item z komórek mezenchymalnych (mięsaki)
    \item tkanka limfatyczna i ukł. krwionośnego (chłoniaki, białaczki)
\end{itemize}
Występują również nowotwory pochodzące z piwerwotnej komórki płciowej.

Nowotwory łagodne - cechy
ostre odgraniczenie guza
rozprężający typ wzrostu
brak zdolności tworzenia przerzutów

Wyjątek stanowią naczyniaki - nowotwór łagodne nie będące otoczone torebką i wciskają się nieregularnymi wypustkami pomiędzy komórki narządu.

\section{Wykład (23.10.2023)}

Radiobiologia - nauka interdyscyplinarna badająca wpływ promieniowania jonizującego na tkankę, układy biologiczne i wyjaśnieniem działania promieniowania.

Herman Muller (odkrycie wywoływania mutacji przez prom X, nagroda nobla w 1946 r.)

Rola radiobiologii klinicznej
\begin{itemize}
    \item poznawanie mechanizmów decydującycch o odpowiedzi nowotwory
    \item opracowanie testów prognostycznych
    \item opracowanie metod leczenia
\end{itemize}

Związki organiczne: białka, węglowodany, dna, woda

Faza fizyczna - wzbudzanie i jonizacja atomów i cząstek amterii

Faza chemiczna - przerwanie wiązań chemicznych, powstanie wolnych rodników

Faza biologiczna - reakcje enzymatyczne (naprawa DNA), śmierć komórek, odczyny popromienne, nowotwory wtórne

Dawka pochłonięta - ilość energii pochłanianej w masie ośrodka [J/kg] = [Gy] (jednostka dawki pochłoniętej, kermy)

$D = \frac{d\epsilon}{dm}$ 

Masowa zdolność hamowania - określa stratę energii wzdłuż toru cząstki [$keV/\mu m$]

$\frac{S}{\rho} = - \frac{dE}{dx} = L = \mathrm{LET}$

Gęstość jonizacji a wymiary DNA

Oddziaływania pośrednie i bezpośrednie

Radioliza wody - w wyniku jnizacji wody tworzy się anionorodnik. Może również powstać aktywny biologicznie elektron uwodniony. Anionorodnik jest niestały i rozpada się na proton wodoru i rodnik hydroksylowy

Tlen jest najsilniejszym chemicznym modyfikatorem działania promieniowania joizującego, ponieważ ilość szkodliwych związków powstających w wyniku radiolizy wody wzrasta wraz ze stężeniem tlenu.

RBE przy dawkach terapeutycznych

Statystyka liczby trafień - opisywany rozkładem Poissona $P(n) = \frac{e^{-x}x^n}{n!}$

Komórkowe mechanizmy obronne przed uszkodzeniami
\begin{itemize}
    \item synteza neutralnych radioprotektorów (enzymów unieszkodliwiających wolne rodniki i inne reaktywne formy tlenu)
    \item naprawa uszkodzeń DNA
    \item eliminacja komórek w przypadku braku możliwości naprawy
\end{itemize}

Możliwe efekty oddziaływania na komórki:
\begin{itemize}
    \item brak efektu
    \item przy małych dawkach: możliwa kancerogeneza, zwiększone prawd. mutacji
    \item przy dużych dawkach: niszczenie komórek
\end{itemize}

Przykłady efektów deterministycznych:
\begin{itemize}
    \item oparzenia skóry
    \item zaćma
    \item sterylizacja
    \item uszkodzenie nerek
    \item ostra choroba popromienna
\end{itemize}

Miara promieniowrażliwości komórek (krzywe przeżywalności)

Model przeżywalności komórek (jednotarczowy, wuelotarczowy, dwuskładnikowy, liniowo-kwadratowy gdzie przeżywalność w zależności od dawki wynosi $S(D) = e^{-\alpha D - \beta D^2}$, gdzie $\alpha$ - stała opisująca pocz. nachylenie krzywej, $\beta$ - odpowiada za składnik kwadratowy)

Przeżywalność komórek

Frakcjonowanie dawki (przyjmuje się, że dawka na frakcję powinna być w zakresie 1,5-2 Gy (wyjątek stanowi terapia raka prostaty), klasycznie stosuje się frakcjonowanie z dawką 2Gy\slash frakcję)

Czas przerwy midzyfrakcyjnych - 6 h dla nowotworu znajdującego się w sąsiedztwie tkanek prawidłowych wcześniej reagujących (wysokie $\alpha / \beta$), 12 - 24 dla nowotworu znajdującego się dalej

4R radioterapii - rozpoznanie 4 podstawowych procesów biologicznych decydujących o odpowiedzi nowotworu i tkanek prawidłowych na frakcjonowaną radioterapię:
\begin{itemize}
    \item Reparacja - naprawianie uszkodzeń radiacyjnych (wymaga czasu t)
    \item Redystrybucja (desynchronizacja - komórki w różnych fazach cyklu) (wymaga czasu t)
    \item Reoksygenacja (wymaga czasu T)
    \item Repopulacja - nadmierny rozrost guza przeciwdziałając niszczeniu nowotworu (wymaga redukcji T)
\end{itemize}

Czas między frakcjami t i całkowity czas terapii T

Parametry:
\begin{itemize}
    \item dawka całkowita
    \item dawka frakcyjna
    \item czas między frakcjami
\end{itemize}

Potencjalny czas podwojenia objętości guza

Biologiczna dawka efektywna

Oddziaływanie promieniowania X z materią:
\begin{itemize}
    \item efekt fotoelektryczny
    \item efekt Comptona
    \item tworzenie par
\end{itemize}

Zależą od energii i rodzaju materiału, w którym propaguje się promieniowanie

Promień klasyczny elektronu $r_0 = 2,818 10^{-13} m$

\section{Wykład (26.10.2023)}

Dlaczego znajomośc oddziaływań promieniowania jest ważne w obrazowaniu?

Jonizacja:

potencjał jonizacji $E_j$

Średnia energia pary jonów W

Dla powietrza $W = 33,97~eV$, dla elektronów $W = 25,2~eV$

Oddziaływania cząstek naładowanych
\begin{itemize}
    \item oddziaływania kulombowskie
    \item zderzenia sprężyste i niesprężyste
\end{itemize}

Maksymalna energia przekazana w zderzeniu

$Q_{max} = 4mME/(M+m)^2$, gdzie E jest energią cząstki padającej.

Ponieważ elektrony są nierozróżnialne, przyjmuje się, że maksymalny transfer energii wynosi 50\%.

Gdy stosujemy mechanikę relatywistyczną, to za masę elektronu podstawiamy jego masę relatywistyczną.

W rzeczywistości zderzenia z elektronami są niesprężyste (następują straty energii).

Zdolność hamowania elektronów: $\frac{dE}{dx} = (\frac{dE}{dx})_c + (\frac{dE}{dx})_r$, składowa zderzeniowa (collision c) i radialna (r).

Elektrony $\delta$ pośredniczą w przekazie energii cząstki naładowanej do absorbującego środowiska. W skórcie cząstka naładowana powoduje powstanie elektronów, które następnie jonizują środowisko oddając tym samym swoje energie.

\subsection{Tor przekazywania energii}

Miarą gęstości jonizacji jest współczynnik LET (linear energy transfer): $LET = dE/dx$

Promieniowanie dzieli się na niskoletowe (X, gamma i el.) i wysokoletowe (alfa, protony, neutrony, ciężkie jądra, mezony $\pi$). ($L_{\Delta} = -(dE/dx)_{\Delta}$, $L_{\infty} = -(dE/dx)_{\infty}$).

\subsection{Wielkości dozymetryczne}

Kerma - suma energii kinetycznej wszystkich cząstek naładowanych uwolnionych przez promieniowanie w masie dm danej substancji. Wliczamy całkowitą energię cząstek wygenerowanych w masie.

Energia przekazana przekazana przez promieniowanie jonizujące materii w danej obj. (suma sum energii, które weszły do obszaru, sum energii, które wyszły z obszaru i sum różnic energii wyzwolonych w przemianach jąder)

Dawka pochłonięta - ilość energii pochłanianej w tkance $D = d\epsilon / dm~[J/kg]$. Wliczamy energię cząstek pozostawionych w masie.

$PDG(g,S)$ - procentowa dawka na głębokości
$PDD(d,S)$ - precentage depth dose

Efektt narostu dawki (build up) - dzięki temu możemy skoncentrować większą ilość dawki na określonej głębokości w tkance.

Pik Bragga (dla protonów)

Przy kermie 

Przy dawce następuje narost. Wynika on z tąd, że przy dawce interesuje nas energia tylko zdeponowana w okręślonej masie tkanki. Dla maksimum następuje równowaga elektronowa (tyle, ile elektronów powstaje, tyle zanika).

\subsection{Terapeutyczne lampy RTG}

Lampa rentgenowska: powierzchniowa, ortonapięciowa

Aparat do telegammaterapii - bomba kobaltowa

Liniowy akcelerator cząstek: fotonowy (6-25 MV), elektronowy (6-30MV)

Urządzenia do radioterapii hadronowej (protony, ciężkie jony, neutrony)

Radioterapia powierzchniowa

Radioterapia ortowoltowa

\subsection{Osłabienie promieniowania X, izodozy}

Linie łączące punkty o tej samej wartości dawki pochłoniętej to izodozy.

HVL - warstwa półchłonna

Bomba kobaltowa i gamma knife

TBI (total body irradiation - mapromienienie calego ciała)

Aktywność właściwa a jest aktywnośćia A danego radionuklidu na jednostkę masy m tego radionuklidu nuklidu.

Równoważna stała ekspozycyjna

Stała $\Gamma$ (air kerma rate constant) - współczynnik przeliczeniowy między aktywnością danego izotopu a ekspozycją

$\Gamma_{AKR}$ (speific air kerma rate constant) - przelicznik aktywności na moc kermy w powietrzu

Metoda izocentryczna - odległość od źródła do izocentrum jest zawsze taka sama.

Głębinowy rozkład dawki.

\section{Wykład (30.10.2023)}

Opóźnienie po włączeniu\slash wyłączeniu

Promieniowanie przez osłonę źródła

Źródła kobaltowe (Co-60)

Jak zbadać szczelność źródła zamkniętego (dokonuje się wymazu z pomocą wacika nasączonego spirytusem i dokonuje się pomiaru dawki. Jeśli dawka nie spełnia warunki dawki dla źródła zamkniętego, to źródło jest nieszczlene, otwarte).

Głowica źródła składa się z:

- metalowej osłony z odprowadzeniem

- system otweiracjący źródło

Wymiana źródła (raz na 5 lat, Co-60 $T_{1/2} = 5,26 \mathrm{roku}$, $E_{\gamma} = 1,25~MeV$)

\subsection{Urządzenia do radioterapii - Przyspieszacze (akceleratory)}

Liniak - akcelerator liniowy (najwyższa energia wiązki - $6~MeV$, średnia energia - $2-3~MeV$)

Elekta

Akcelerator - służy do kontrolowanego przyspieszania cząsteczek, przyspieszenie pod wpływem pola elektrycznego (tylko cząstki naładowane), do skupienia wiązki używa się odpowiednio ukształtowanego pola magnetycznego (odchodzi się od wiązek elektronowych)

Monoenergetyczne wiązki elenów (4 MeV) do 25 MeV
Wiązki fotonów - otrzymywane przez konwersję wiązki elektronowej na promieniowanie hamowania, wiązki promieniowania X charakteryzują się ciągłym widmem energetycznym ograniczonym energią elektronów inicjujących

Co jest potrzebne?

Źródło cząstek naładowanych

Elementy przyspieszające

Elementy transportujące

Systemy towarzyszące (kształtowanie i monitorowanie wiązki, układ chłodzenia, system podtrzymywania próżni, osłony)

Varian Clinac 2300 DC

Schemat budowy akceleratora:

\begin{itemize}
    \item źródło elektronów
    \item struktura przyspieszająca
    \item magnes zakrzywiajacy
    \item głowica akceleratora
    \item modulator impulsowy
    \item magnetron lub klistron
    \item klimatory pierwotny i szczękowe
    \item filtry i osłony
\end{itemize}

Akceleratory niskich energii - tarcza umieszczona jest w sekcji przyspieszającej, a więc system transportu wiązki elektronów między falowodem a tarczą

Element przyspieszający (najcześciej wykonany z miedzi)

Działo elektronowe jako źródło elektronów, el. emitowane są termicznie z rozgrzanej katody, zogniskowane w wąską wiązkę i przyspieszone w kierunku anody z otworem. O przyspieszeniu cząstek decyduje różnica potencjałów między pkt. początkowym i końcowym

Struktury przyspieszające (falowód), jest strukturą metalową o przekroju kwadratu lub koła, służącą do transmisji mikrofal.

Elektorny są niesione przez szczyt fali radiowej.

Struktura czasowa wiązki: wiązka jest impulsowa, typowe parametry impulsu pokazano na rysunku (wysokośc im. 50 mA, długość 2 mikrosenkundy)

Systemy wspomagające:
\begin{itemize}
    \item układ pomp próźniowych
    \item układ wodny chłodzenia
    \item układ pneumatyczny do poruszania terapeutycznych
    \item układ osłon pochłaniających promieniowanie promieniotwórcze
\end{itemize}

Akceleratory średnich i wysokich energii

Odchylenie achromatyczne

Podstawowe elementy akceleratora:
ramię
podstawa ramienia
konsola kontrolna

Napromienienia kilkoma wiązkami (rozdzielenie dawki promieniowania na kilka wiązek, przez co nie narażamy tkanek zdrowych)

Ustawienia akceleratora:

Ustawienie izocentryczne

\subsection{Formowanie wiązek promieniowania X}

- wiązka niemal monoenergetyczna
- przekrój wiązki w kształcie elipsy
- szerokość połówkowa 0,5 - 3,4 mm

Folie rozpraszające (dwa stopnie rozpraszania: pierwotna i wtórna, folia pierwotna z materiału o dużym Z, folia wtórna z materiału o małym Z)

Głowice terapeutyczne

Tarcza konwersji (transmisyjne tarcze konwersji)

Tarcza "cienka":
- wyższa średnia energia wiązki fotonowej
- skażenie strumieniem elektronów
- mała wydajność konwersji

Tarcza "gruba":
- niższa energia wiązki fotonowej
- brak skażenia strumieniem elektronów
- wyższa wydajność konwersji

Energia promieniowania hamowania
- widmo ciągłe
- widmo charakterystyczne

Formowanie wiązek fotonowych

Geometria promieniowania
Podstawowe parametry:
\begin{itemize}
    \item wartość obrotu głowicy aparatu terapeutycznego wokół pacjenta
    \item pozycja stołu operacyjnego
    \item wartość obrotu kolimatora
    \item wartość pól terapeutycznych wiązki promieniowania
\end{itemize}

Wymiary prostokąta kolimatora $P(F) = P(F_I)*F/F_I$, $I$ - punkt izocentryczny, $F_I$ - odległość izocentryczna

Filtr wyrównujący osłabia promieniowanie w centralnej części wiązki redukując jej intensywność do poziomu z brzegów pola.

Filtr wyrównujący

Kolimator wstępny (umiejscowiony w pobliżu tarczy)

Transmisyjna komora jonizacyjna

- najpopularniejsze detektory promieniowania dla akceleratorów liniowych

Dwie komory (dwie niezależne komory pomiarowe, w razie awarii jednej komory zawsze można skorzystać z drugiej, podwaja pomiar - zapewnia prawidłowe określenie dawki)

Wymagania odnośnie komór jonizacyjnych

Profile wiązek megawoltowych:
- obszar centralny - profil płaski, w granicach 80\%
- połcień - obszar między 80\% a 20\% dawki osi
- cień - na ogół do obszaru gdzie jest poniżej 1\% dawki w osi

Jednorodność wiązki - wiązkę promieniowania X charakteryzuje duża niejednorodność

Umbra i penumbra (profil wiązki)

Stosunek dawek na osi i 20 cm do niej wynosi

Narzędzia do modyfikacji wiązki fotonowej:
- bloki (były wykonywane ze stopów metali, ograniczają obszar dużych dawek do zaplanowanego targetu - stałe lub indywidualne) - obecnie już nie używane, obecnie częściowo zastąpione przez kolimatory wielolistkowe
- kliny
- wielolistkowe

Osłony indywidualne

Ukośne wejście wiązki

Wiązka klinowa

Modyfikacja wiązki filtrami klinowymi

Pola klinowane

Kompensatory: od kompensacji, do modulacji (IMRT)

Umiejscowienie MLC

\section{Wykład (06.11.2023)}

\subsection{Fizyczne parametry wiązki fotonowej}

- Rozmiar pola napromieniania A

- Głębokość terapeutyczna z - 10 cm dla Megawoltowych

- odległość SSD i SAD (Source-Axis Distance - przy technice izocentrycznej)

- Energia wiązki fotonów

- Liczba wiązek użyta podczas terapii

- Czas napromienienia - związana z wydajnością akceleatora i z dynamiką podawania wiązki.

Fantom - objętość materiału tkankopodobnego, o wymiarach na tyle dużych aby zapewnić warunki pełnego rozproszenia stosowanej wiązki promieniowania.

Zwykle materiały odpowiadające tkance miękkiej (woda), choć czasem stosuje się fantomy anatomiczne przypominające kształtem i gęstością do ciała pacjenta.

Fantomy składane z warstw, czasami z otworami na detektory termoluminescencyjne.

Fantom wodny - woda jako standardowy materiał, zarówno dla wiązek fotonów i elektronów

Stosuje się również fantomy stałe - musi imitować wodę pod wzgl. gęstości, liczby el. w gtamie materiału, efektywnej liczbie atomowej $Z_{eff}$.

Dla promieniowania $Z_{eff} = \sum_{i}^{} (a_i Z_i^2 / A) / \sum_{i}^{} (a_i Z_i / A)$

Najważniejsze parametry do wyznaczania dawki promieniowania zewn.

funkcje dozymetryczne stosowane dla całego zakresu
\begin{itemize}
    \item \% dawka głęboka (PDG)
    \item współczynniki uwzględniające wielkość pola napromienienia
    \item profil wiązki
\end{itemize}

funkcje dozymetryczne stosowane dla Co-60 i niższych energii

Geometria standardowa

Musimy wyznaczyć punkt pomiarowy na jakiejś głębokości. Głębokość związana jest z energią promieniowania. Z punktu widzenia terapii metoda polega na wyznaczeniu rozkładu dawki w geometrii standardowej a następnie ten rozkład jest korygowany a możliwie najlepiej opisywał rzeczywistość.

$D_{rz}(x,y,z) = D_w(x,y,z)\prod_{n=1}^{N}k_n$, gdzie kn to współcz. poprawkowy uwzględniający n-ty efekt.

Do obliczania współcz. wprowadza się specjalne parametry charakteryzujące wiązkę promieniowania.

Stosuje się też metody Monte Carlo

W IMRT tzw. planowanie odwrotne

Poziom wydajności aparatu terapeutycznego (technika SSD):

Wydajność mierzymy w fantomie wodnym lub stałym, głębokość w zależności od energii będzie 10 cm lub 5 cm. Odległość źródła od powierzchni fantomu taka jak...

Jednostki monitorowe (MU - monitor units):

Wydajność aparatu terapeutycznego - zależność między wskazaniami komory minitorowej wyrażonej w MU a dawką promieniowania zmierzoną w fantomie wodnym w warunkach referencyjnym. Definiuje się ją jako zmierzoną w wfantomie wodnym wartość dawki prom. wutworzonego przez dany aparat terapeutyczny w ściśle określonych warunkach referencyjnych.

Pole z filtrem klinowym - oś komory jonizacyjnej skierowana równolegle do krawędzi przecięcia płaszyzn tworzących klin.

$MD_{STklin} = W \cdot MD_{ST}$

Określanie mocy dawki w fantomie w osi wiązki:

$MD(g,S) = \frac{MD_{ST}\times q(S)}{PDG(5,S)}PDG(g,S)$ dla Co-60

$MD(g,S) = \frac{MD_{ST}\times q(S)}{PDG(10,S)}PDG(g,S)$ dla energii megawoltowych

Pola prostokątne i kołowe

$AB/[2(A+B)] = A_{eq}^2/(4A_{eq})$

$\pi r^2 = A_{eq}^2$

Pole napromienienia geometryczne i dozymetryczne

Fantom wodny - oznaczenia (metoda SSD)

Procentowa dawka głęboka

Technika SSD. Peak scatter factor (PSF): $PSF(A,h\nu) = \frac{D_p(z_{max}, A, f, h\nu)}{D_p(A, h\nu)}$

Wpływ rozproszeń w fantomie na dawkę w osi centralnej

Zaleśność od wielkości pola i od SSD

Składowe dawki w fantomie

Pole zmodyfikowane bloczkiem

Określanie mocy dawki w fantomie w osi wiązki: $R(s,d) = PDD(s,d) \times SF(s) \times OF(s) \times X \times W$

Technika izocentryczna

Tissue Air Ratio ($TAR(s,d)$) - iloraz wartości mocy dawek zmierzonych w osi wiązki w fantomie na danej głębokości d i w powietrzu w warunkach równowagi elektronów.

Związek TAR i PSF

Tissue Phantom Ratio (TPR) - iloraz mody dawki zmierzonej na danej głębokości d oraz na głębokościach 5 lub 10 cm

Określanie mocy dawki w fantomie w osi wiązki: moc dawki w izocentrum dla danej jakości promieniowania - $MD(d,S) = MD_{ST} \times c(S) \times TPR(d,S)$, c uwzględnia wielkość pola w izocentrum

Tissue Maximum Ratio (TMR) - iloraz wartości mocy dawek mierzonych w fantomie w osi wiązki na danej głębokości i na głębokości maksymalnej dawki.

Profil wiązki i korekty

Fotoneutrony

\subsection{Oddziaływanie neutronów z materią}

Odziaływują pośrednio

W zależności od energii neutrony dzielimy na: termiczne (0-0.1 eV), powolne, pośrednie, prędkie, wielkich energii

Detekcja przez wychwyt na borze-10.

Zatrzymywanie neutornów:
Parafina, woda

Oddziałwyanie z materią: rozpraszanie sprężyste lub niesprężyste, pochłanianie (wychwyt (radiacyjny lub z emisją) lub reakcje jądrowe (produkty naładowane lub produkty neutralne))

Wtórne cząśtki naładowane (neutrony powodują znacznie wieksze skutki bilogiczne niż promieniowanie gamma).

Aktywacja neutronowa

System wielkości dozymetrycznych

Dawka efektywna - suma dawek równoważnych od promieniowania zewn. i wewn. w najważniejszych tkankach i narządach.

Przestrzenny równoważnik dawki $H^*(10) = D^*(10) \times Q^*(10)$

Procesy generacji neutornów: reakcje elektron-neutron (e,n), gamma-neutron ($\gamma$,n)

Generowane są w otoczeniu liniowych akceleratorów medycznych w powyżej wymieniownych reakcjach przy energiach pow. 6 MeV

Radioterapia z modulowaną intensywnością (IMRT)

Osłony: pierwsza warstwa służy do spowolnienia neutronów, następnie skuteznie pochłaniane przez B-10 lub kadm. Bor w postaci węglika boru B4C lub Boralu.

\subsection{Obrazowanie w radioterapii}

Rola obrazowania w radioterapii:
\begin{itemize}
    \item przy diagnozie, decyzji o terapii
    \item obrazowanie przy planowaniu terapii
    \item weryfikacja planu leczenia
    \item weryfikacja poprawności leczenia
    \item kontrola diagnostyczna
\end{itemize}

Techniki obrazowania:
\begin{itemize}
    \item rentgenowaska tomografia komputerowa (CT)
    \item magnetyczny rezonans jądrowy
    \item PET-CT
\end{itemize}

Skala Housfielda, HU

$H_{tkanki} = 1000 \frac{\mu_{tkanki}-\mu_{H_2O}}{\mu_{H_2O}}$

Obrazowanie do radioterapii

Symulatory (radioterapeutyczne lub tomograficzne)

Nowoczesne symulatory służą do:
\begin{itemize}
    \item lokalizacji tkanek nowotw. i prawidłowych
    \item symulacji radioterapii
    \item weryfikacji planu leczenia
    \item monitorowania leczenia
\end{itemize}

Symulator CT
\begin{itemize}
    \item płaski balt stołu
    \item duży otwór pierścienia CT
    \item system wskaźników laserowych
    \item wirtualny symulator
\end{itemize}

Wirtualna symulacja:

Zalety
\begin{itemize}
    \item eliminacja dodatkowego kroku
    \item bez pacjenta
    \item bez dodatkowego urządzenia
    \item oróbka - łączenie obrazów
    \item jako obraz odniesienia do dalszej kontroli - bliższy oryginału w oparciu o który wykonano plan
\end{itemize}
Wady
\begin{itemize}
    \item obraz zamrożony
    \item dużo danych (szybkość vs jakość)
    \item ograniczenie wielkości otworu
    \item pacjent nie przyzwyczaja się do aparatu
\end{itemize}

Wyznaczanie gęstości elektronowej z HU

Obrazowanie portalowe EPID

\section{Wykład (13.11.2023)}

\subsection{Medycyna nuklearna}

Ustawa o prawie atomowym

Medycyna nuklearna - działalność diagnostycza związana z podaniem pacjentom produktów radiofarmaceutycznych, a także leczenie polegające na zamierzonym wprowadzaniu do ustroju terapeutycznych ilości produktów radiofarmaceutycznych.

Gammakamera - obrazowanie ilościowe

Terapia - badanie ilościowe

Techniki obrazowania medycznego: z użyciem promieniowania jonizującego (rentgenodiagnostyka, medycyna nuklearna), bez użycia promieniowania jonizującego (MRI, USG, metody optyczne)

Obrazowanie w:
radiologii - transmisyjne (akt zarejestrowania obrazu - ekspozyja)
medycyna nuklearna - emisyjne (akt zarejestrowania obrazu - akwizycja)

Techniki obrazowania: strukturalne (jak wyglądają struktury ciała, tomografia komputerowa, rezonans magnetyczny) i funkcjonalne (jak działają dane narządy, scyntygrafia, funkcjonalny rezonans magnetyczny)

Medycyna nuklearna stosuje obrazowanie funkcjonalne

Medycyna nuklearna używa otwartych źródeł promieniotwórczych. Najczęściej w formie płynnej. Wydzielone miejsca poruszania się, miejsca dla personelu i dla pacjentów osobno. Specjalne procedury, zalecenia dla pacjentów po przeprowadzonym badaniu. 

Poniżej 800 MBq (40 uS / h) - granica

Pola działania medycyny nuklearnej i ich potencjał:

Terapia izotopowa
obrazowanie molekularne
topograficzna i czynnościowa diagnostyka radioizotopowa
diagnostyka radioimmunologiczna in vitro

Promieniowanie:
- bardzo krótki zasięg działania
- terapia radem (badania z wykorzystaniem promieniowania alfa, coraz więcej badań dotyczących wykorzystania tego typu promieniowania w radioterapii)

Co do terapii: alfa i beta
Co do diagnostyki: gamma (ma nieskończony zasięg)

Stosowane izotopy: diagnostyka (Tc-99m, I-123, I-131, F-18, Ga-68), terapia (Lu177, Re186, Y90, Sm153, Ra223, As211, In111)

Aktywności rzędu MBq, a nawet setek MBq. Czasem rzędu GBq.

Najważniejsze cechy źródła: aktywność, czas połowicznego rozpadu, typ rozpadu, głowne (wykorzystywane) linie energetyczne, typ źródła (otwarte/zamknięte, płyn/ciało stałe)

Oddziaływanie promieniowania X, gamma - reakcja fotojądrowa

90\% badań scyntygraficznych z wykorzystaniem Tc-99m (wytwarzanie przy użyciu generatorów technetu)  aktywność 8-10 GBq, czas 1/2 - 6,02 h, typ rozpadu - gamma, linie energetyczne - 140 keV (monoenergetyczny, optymalna do obrazowania), źródło otwarte

Podstawy obrazowania w medycynie nuklearnej

- gammakamera (1, 2, lub 3-głowicowe) - rozdzielczość zależy od odległości głowicy od pacjenta (aby kolimator zatrzymał jak najwięcej rozprzoszonych kwantów)

Zadania fizyka medycznego w medycynie nuklearnej (scyntygrafia)
- Sporządzanie protokołów akwizycyjnych
- planowanie pomiarów akwizycji

Budowa pozytonowego tomografu emisyjnego (PET) - nie ma fizycznego kolimatora (korzysta się z kolimacji elektroniczna) - wykorzystuje się radiofarmaceutyki emitujące promieniowanie beta+, pozytony anihulują z elektornami, w wyniku czego powstają dwa kwanty gamma o energiach 511 keV poruszających się w przeciwnych kierunkach. Detektory rejestrują te kwanty i określają tor ruchu tych fotonów (udział koincydencji prawidłowej - 30\%). W badaniach PET nie da się przeprowadzać badań dynamicznych (wynika to z małego zakresu widzenia: ok. 30 cm, w trakcie badania całego ciała wykonuje się obrazowanie po fragment po fragmencie). Elementy detekcyjne plastikowe - niska wydajność detekcji, ale ich praca jest szybka (krótki czas zaniku scyntylacji - 2 ns).

Parametry wpływające na rozdzielczość: zasięg pozytonów - zależny od spektrum energetycznego rozpadu, niekolinearność pozytonów - ok. 0,5 stopnia różnicy, związana z resztkową energią pozytonu, rozmiar detektora - proporcjonalne do d/2

Izotopy: fluor-18 (cyklotronowy), miedź-62/64, gal-66/68, węgiel-11, azot-13

Obecnie sprzedawane są systemy PET+CT

Zastosowania kliniczne scyntygrafii i PET (choroby nowotworowe, układu nerwowego, serca)

Koincydencja prawidłowa, przypadkowa, pochodzące z rozproszeń


\section{Wykład 20.11.2023}

Zależność piku braga od dawki promieniowania.

Rozproszenie wiązki (w wąskich wiązkach pik Bragga staje się słabiej uwidoczniony)

Przeżywalność komórek

Terapia Baronowo-neutronowa

Dozymetria Kliniczna

\subsection{Brachyterapia}

Metody brachyterapii
\begin{itemize}
    \item Metoda LDR
    \item Metoda HDR
    \item Brachyterapia śródoperacyjna
    \item Brachyterapia okołooperacyjna
    \item Metoda PDR
\end{itemize}

Źróła promieniotwórcze

Moduły powierzchniowe

Implanty wewnątrzjamowe

Implanty wewnątrztkankowe

Implanty długotrwałe

Stenty

\section{Wykład (27.11.2023)}

Obszary w radioterapii

obszar guza

Kliniczny obszar napromienienia

zaplanowany obszar napromienienia

obszar leczony

obszar napromieniony

Obszar guza GTV - to masa guza, którą daje się ustalić istniejącymi metodami diagnostycznymi

Obszar kliniczny napieniania CTV - to powiększony obszar GTV. Powiększany jest o pewien margines ze względu na większe prawdopodobieństwo występowania komórek nowotworowych wokół guza.

Marginesy wokół CTV:

\begin{itemize}
    \item marginesy wewnętrzne - wynikają z ruchu narządów
    \item marginesy zewn - wynikają z ułożenia pacjenta i geometrii wiązki
\end{itemize}

Zaplanowany obszar napromieniania - PTV - zawiera obszar ITV (margines wewn.) oraz dodatkowy margines uwzględniający niepewność pozycjonowania, dokładność ustawień akceleratora i inne czynniki zewnętrzne. CTV plus stały lub zmienny margines.

Obszar leczony - obszar objęty pewną, powierzchnią izodozową (np. 90\%).

Narząd krytyczny (Organ at risk OAR) - leżący w pobliżu PTV narząd, w którym dawka wynikająca z planu terapii jest bliska wartości tolerowanej. Pojawienie się takiego obszaru może spowodować konieczność modyfikacji.

Narządy radiowrażliwe - np. soczewki oczu, żuchwa, ślinianki. Narządy, których wrażliwość zależy od sposobu frakcjonowania dawki powinny być omijane.

Uwzględnienie półcienia - należyt odpowiednio dobrać obszar napromieniania. Musimy zaplanować większe pole aby cały obszar PTV objąć izodozą 90\%.

Immobilizacja pacjenta - środki immobilizacji obejmują taśmy, maski (np. siatkowe, na całe ciało), ramki mocowane do czaszki pacjenta, gryzaki, formy (ułożenie pod pacjentem poduchy z drobnych kulek piankowych). Podkładki pod głowę (najprostszy środek immobilizacji).

Ramka stereotaktyczna - w radiochirurgii, mocuje się śrubami do czaszki pacjenta.

\subsection{Planowanie leczenia}

Podstawowy dogmat radioterapii - dostarczyć dawkę zaplanowaną do nowotworu i zrobić wszystko, aby oszczędzić tkanki zdrowe.

Teleradoterapia - prowadzi się na ogół w zykorzystaniem wielu wiązek promieniowania o różnych energiach, padających z różnych kierunków.

Planowanie leczenia (fizyk musi zrozumieć, co dzieje się w danym systemie)

Zlecenie terapii i dokumentacja - może zależeć od dostępności aparatury, dokumentacja ujednolicona i przejrzysta, umożliwia odpowiednio wykształconej osobie zrozumieć, co się działo z pacjentem.

Optymalizacja rozkładu dawki: $P_+ = PMW(D_{ijk}) \cdot \prod_{l=1}^{L}(1-PWM_l(D_{ijk}))^{sk}$

$PMW(N_0, d, D) = e^{-N_0exp(-\alpha D + \beta D d)}$

Znormalizowany gradient dawki $\gamma = \frac{dPMW}{dD}|_{D_{0}}D_0$. w praktyce obserwujemy słabszą zależność pomiędzy dawką...

Kryterium jednorodności rozkładu dawki (uśrednianie dawki)

PWU - jednostki funkcjonalne narządów - prawdopodobieństwo wystąpnienia uszkodzenia.

Jednostki funkcjonalne połączone szeregowo - uszkodzenie jednej z nich powoduje upośledzenie funkcji całego narządu - dawka nawet w niewielkeij obj narządu nie przekracza pewnej wartości granicznej.

PWU - zależność od objętości (Model Lymana)

$TD_{50}$ - dawka, której podanie w obj V pwoduje wystąpienie uszkodzeniau 50\% pacjentów.

Cel planowania leczenia - określenie takiej geometrii wiązek aby przy z góry zadanej dawce całkowitej aplikowanej w PTV podać jak najmniejszą dawkę w obszarze narządów szczególnie wrażliwych na promieniowanie.

Punkt referencyjny

Histogram dawka-objętość (DVH) - we współczesnej radioterapii określenie dawki pochłoniętej dla odpowiednich objętości, a nie dal pojedynczych punktów, ma kluczowe znaczenie w jej przypisywaniu, obliczaniu i raportowaniu, ponieważ.

DVH - wyrażane jako objętość bezwzgl. lub objętość w stosunku do całkowitej objętości struktury, otrzymujące co najmniej daną dawkę pochłoniętą D.

Dawki prawie minimalna i prawie maksymalna ($D_{100\%}$ potocznie nazywa się dawką minimalną (pochłonięta w jednym lub kilku wokselach)i odnosi się nie do wartości dawek, a do objętości, $D_{98\%}$ to dawka prawie minimalna, $D_{2\%}$ to dawka prawie maksymalna) - \textbf{na kolokwium (opisać histogram)}

Planowanie leczenia - wprost i odwrotne (\textbf{Na kolokwium})

Wprost:
\begin{itemize}
    \item Użytkownik określa parametry wiązek
    \item TPS oblicza rozkład dawki 3D
    \item jeżeli wynik niesatysfakcjonujący, użytkownik modyfikuje parametry wiązek
\end{itemize}

Odwrotne:
\begin{itemize}
    \item Określamy, co chcemy uzyskać na końcu
    \item TPS wyznacza metodą iteracyjną profil fluencji wiązek (wyznacza parametry)
    \item jeżeli wynik niesatysffakcjonujący, użytkownik modyfikuje kryteria
\end{itemize}

Planowanie leczenia w terapii klasycznymi akceleratorami oraz wiązkami stacjonarnymi

Metody SSD oraz SAD izocentryczna

Rożne pola napromieniania (małe pole, standardowe pola o nieregularnym kształcie, bardzo duże pola napromieniania)

Rola symulacji - wykrzystuje się w procesie radioterapii do zebrania danych anatomicznych pacjenta, do weryfikacji czy plan może buć realizowany.

Cyfrowo rekonstruowane radiogramy (DRR)

Planowanie 1D, 2D, 3D, 4D

Zasady doboru liczby wiązek - najprostszą techniką wielopolową jest napromienienie dwoma wiązkami. Dla takiej techniki równanie nie ma rozwiązania. W tym przypadku ososba planująca. Zwiąkszenie liczby wiązek przekłada się na wydłużenie redioterapii.

Po zastosoaniu klinów minimalizujemy różnice w ułożeniu wiązek (w technice dwóch wiązek).

Warunki jednorodności

\subsection{Planowanie radioterapii}

planowanie 3DCRT

IMRT - intensity modulated radiotherapy (dwie podstawowe metody: dynamiczna (posuszanie się listków w trakcie napromieniania), sekwencyjna (każde pole podzielone jest na szereg mniejszych segmentów, które są napromieniane sekwencyjnie))

Klasyczne planowanie leczenia - wprost i odwrotne

IMRT - złożony proces

Na czym polega planowanie odwrotne? - podział wiązki na mikrowiązki (beamleat)

Obliczanie dawek dla IMRT $D_i = \sum_{j=1}^{N}C_{ij}W_j$

Symulowane wyrzażanie - tak dobrać parametry, żeby uzyskać maksymalnie z wykorzystaniem dostępnych narzędzi odpowiedni wynik

Proces planowania IMRT
\begin{itemize}
    \item wyznaczanie konturów GTV, CTV, PTV przez lekarza
    \item fizyk wybiera położenie izocentrum, liczbę wiązek, kierunki wejścia wiązek
    \item wprowadza cele optymalizacji i inicjuje iteracyjny proces optymalizacji
    \item wygenerowany plan leczenia podlega ocenie pod kątem zakładanych celów. Jeżeli zaakceptowany, proces się kończy. Jeżeli nie, dobór parametrów i optymalizacja są powtarzane.
\end{itemize}

Ogony izodozy dawki (60\%) - nie powinny one występować

Powszechne kompromisy

Następny wykład 11.12 (04.12 nie ma wykładu)

\end{document}