\documentclass{article}
\usepackage{polski}

\usepackage{hyperref}
\usepackage[margin=2.5cm]{geometry}

\setlength{\parindent}{0pt}

\title{Wstęp do Fizyki Medycznej - wykłady}
\author{Maciej Standerski}
\begin{document}
\maketitle

\section{Wykład (09.10.2023)}

\subsection{Wstęp}

Email: piotr.tulik@pw.edu.pl

Zaliczenie kolokwium (test: forma zamknięta i otwarta)

Maksymalna liczba do zdobycia na kolokwium: 30

Zaliczenie od 51\% punktów z kolokwium.

Literatura:

"Planowanie leczenia i duzymetria w radioterapii", J.Malicki, K. Ślisarek

"Biocybernetyka i inżynieria biomedyczna 2000" i "Fizyka medyczna", Maciej Nałęcz

"Fizyczne metody diagnostyki medycznej i terapii", A Hrynkiewicz, E. Rokita

"Dozymetria promieniowania jonizującego w radioterapii" i "Podstawy fizyki promieniowania jonizującego..." W. Łobodziec

\subsection{Czym jest fizyka medyczna}

Jakie dziedziny obejmuje fizyka medyczna:
\begin{itemize}
    \item diagnostyka
    \item radioterapia (teleradioterapia i brachyterapia)
    \item medycyna nuklearna (terapia izotopowa, SPECT, itd.)
    \item inżynieria biomedyczna
    \item fizykoterapia (laseroterapia, elektroterapia, galwanizacja, itd.)
    \item kontrola jakości
    \item steryzlizacja radiacyjna
    \item ochrona radiologiczna
    \item zjawiska elektromagnetyczne w diagnostyce i tomgrafia MR
\end{itemize}

Tomografia (PET, SPECT, tomografia rezonansu magnetycznego)

\subsection{Radioterapia}

Radioterapia - miejscowa metoda leczenia nowotworów, wykorzystujące energię promieniowania jonizującego. Stosowana w onkologii do leczenia chorób nowotworowych oraz łagodzenia bólu i innych dolegliwości związanych z rozsianym procesem nowotworowym, np. w przrzutach nowotworowych do kości lub guzach powodujących niedrożność oskrzela.

Promieniowanie można podzielić ze względu na:

\textbf{Oddziaływanie z materią}
\begin{itemize}
    \item Promieniowanie pośrednio jonizujące elektromagnetyczne: X i gamma (rozróżniamy na podstawie źródła pochodzenia: promieniowanie $\gamma$ jest promieniowaniem rentgenowskim powstającym podczas przemian jądrowych, natomiast promieniowanie X jest promieniowaniem emitowanym przez cząstkę poruszającą się ruchem przyspieszonum)
    \item Promieniowanie cząstkowe (bezpośrednio jonizujące, pośrednio jonizujące (neutrony))
\end{itemize}

\textbf{Energię}
\begin{itemize}
    \item Radioterapia konwencjonalna (60 d0 400 keV)
    \item Radioterapia megawoltowa (1,25 do 25 MeV)
    \item Elektrony (6 do 22 MeV)
    \item Wiązki hadronowe (60 do 230 MeV)
\end{itemize}

Zalety megawoltowego promieniowania X
\begin{itemize}
    \item większa przenikliwość
    \item mniejsza zdolność pochłaniania przez tkankę kostną
    \item lepsza tolerancja leczenia
\end{itemize}

Brachyterapia - umieszczenie źródła promieniowania w jamach ciała, bezpośrednio w guzie albo w jego otoczeniu, najczęściej wykorzystywanym radiofarmaceutykiem jest izotop irydu 192 (źródło zamknięte lub źróło otwarte). Rozróżnia się brachyterapię:
\begin{itemize}
    \item wewnątrzkomórkową - umieszczenie źródła w guzie
    \item wewnątrzjamową - umieszczenie źródła w bezpośrednim sąsiedztwie guza przy użyciu naturalnych otworów w ciele
    \item powierzchniową - umieszczenie źródła na skórze w celu leczenia zmian powierzchniowych
    \item śródnaczyniową - źródła umieszczone w naczyniach krwionośnych
    \item śródoperacyjną
\end{itemize}

Teleradioterapia - technika leczenia za pomocą promieniowania jonizującego (radioterapia), w metodzie tej źródło promieniowania umieszczone jest w pewnej odległości od tkanek. Polega na napromienianiu wiązkami zewnętrznymi określonej objętości tkanek, obejmującej guz nowotworowy z adekwatnym marginesem tkanek oraz, w razie potrzeby, regionalne węzły chłonne. Jednym z rodzajów terapii jest terapia hadronowa, w której wyróżnia się:
\begin{itemize}
    \item terapię cząstkami naładowanymi takimi jak protony, jony, ujemne mezony pi
    \item terapię cząstkami neytralnymi, czyli neutronami, w której wyróżnia się natomiast FNT (terapia szybkimi neutronami) oraz BNCT (terapia borowo-neutronowa)
\end{itemize}

Terapia izotopowa - podanie izotopu promieniotwórczego (najczęściej jod, w diagnostyce glukoza jako nośnik)

Skutki promieniowania: deterministyczne (gdy zostanie przekroczony próg) i stochastyczne (występują zawsze)

Frakcjonowanie - podział na mniejsze dawki

\subsection{Historia promieniotwórczości}

Bomba radowa 

Kliniczny akcelerator van de Graaff'a

Bomba kobaltowa - urządzenie do teleradioterapii lub napromieniowywania przedmiotów promieniami gamma ($\gamma$) o energiach 1,17 i 1,33 MeV, emitowanymi przez izotop kobaltu 60Co o aktywności rzędu 1013-1014 Bq. Ze względu na dużą przenikliwość promieniowania gamma aktywny kobalt jest otoczony grubą osłoną z ołowiu, w której znajdują się kanały wyprowadzające na zewnątrz wiązkę promieniowania. Bomba kobaltowa może też być wyposażona w mechanizm umożliwiający zdalną manipulację próbkami bez narażania operatora na promieniowanie. Bomba kobaltowa jest stosowana w lecznictwie do zwalczania chorób nowotworowych, w defektoskopii, do sterylizacji żywności oraz w chemii radiacyjnej, do badań procesów fizykochemicznych zachodzących podczas napromieniowywania wysokoenergetycznymi kwantami gamma prostych i złożonych układów chemicznych. Nazwą tą określana jest także broń jądrowa z płaszczem kobaltowym.

Insytuty medycyny nuklearnej w Polsce:

1932 - otwarcie Instytutu Radowego w Warszawie

Instytut Onkologii im Marii Skłodowskiej Curie

Narodowy Insytut Onkologii im. Marii Skłodowskiej Curie Instytut Badawczy

Cyklotronowe Centrum Bronowice IFJ PAN (cyklotron AIC-144, Cyklotron Proteus C-235)

Obrazowanie planarne: radiografia ogólna, fluoroskopia, mammografia, stomatologia, densytometria

Scyntygrafia

\subsection{Lampa rentgenowska}

Anoda posiada wolframową powierchnię (posiada wysokoą liczbę atomową $Z = 74$ oraz wysokoą temperaturę topnienia i niski wskaźnik parowania) wtopioną w miedzianą tarczę. Powierzchnia znajdująca się na anodzie może być również wykonana z renu, a w przypadku lamp stosowanych w mammografii może być również wykonana z molibdenu ($Z=42$) ze względu na odpowiednią energię powstającą w wyniku zderzeń elektronów z tarczą anody.

Lampy rentgenowskie ze stałą oraz z wirującą anodą

$N = cT^2e^{-dT}$

Całkowita energia promieniowania rentgenowskiego

$W = kZE_0^2$

Lampy mogą być wolnoobrotowe lub szybkoobrotowe.

Na żywotność lampy zasadniczy wpływ mają łożyska (przy używaniu lampy należy oszczędzać łożyska)

Anoda stacjonarna (powierzchnia rzędu $4~mm^2$)

Anoda wirująca ($1835~mm^2$, wynikiem stosowanie tego rodzaju lampy jest większe liczba fotonów emitowanych z powierchni i dzięki tamu samym krótszy czasu ekspozycji. Dzięki temu, że lampa nie jest bombardowana przez elektrony tylko w okolicach jednego punktu, tylko na całej powierzchni anody, znacznie wolniej się ona nagrzewa.)

Ognisko elektryczne, rzeczywiste i optyczne (pozorne) (im większe ognisko, tym gorsza rozdzielczość obrazu)

Kąt nachylenia anody

Szklana obudowa - utrzymanie próżni $10^-6 mmHg$, odizolowanie elektrod, zespolenie katody i anody.

Kołpak ochronny - chroni przed wydostaniem się promieniowania w niepożądanym kierunku. (olej transformatorowy)

Moc lampy 

Co wpływa na uszkodzenie lampy:
\begin{itemize}
    \item Zbyt długi czas ekspozycji
    \item Zbyt krótki czas pomiędzy ekspozycjami
    \item Odparowanie katody
\end{itemize}

Typy aparatów rentgenowskich:
\begin{itemize}
    \item Aparat typu głowicowego (zasilacz wysokiego napięcia + lampa RTG w kołpaku)
    \item Aparat typu kołpakowego (zasilacz stanowi oddzielne urządzenie)
\end{itemize}

Zasilacze wykorzystywane przy aparatach rentgenowskich:
\begin{itemize}
    \item jednopulsowe
    \item dwupulsowe
    \item sześciopulsowe
    \item dwunastopulsowe
\end{itemize}

np. Zasilacze impulsowe WCZ

Rodzaje ograniczników:
\begin{itemize}
    \item stałe
    \item nastawne
    \item głębinowe
    \item irysowe
    \item uciskowe
\end{itemize}

Stopień nieostrości geometrycznej $n = \frac{s*p}{f-p}$

Efekt półcienia - powstaje, gdy duże ognisko znajduje się w małej odległości od obrazowanego obiektu. Można go zminimalizować oddalając ognisko.

Czynniki wpływające na jakość zdjęcia:
\begin{itemize}
    \item dobór warunków ekspozycji
    \item wielkość ogniska lampy rentgenowskiej (im mniejsze ognisko tym lepsza zdolność rozdzielcza)
    \item wartość i rodzaj zastosowanej filtracji całkowitej (dodatkowa filtracja powoduje zmniejszenie dawki, którą pacjent pochłonie (można ograniczyć ilość miękkiego promieniowania, która nie dodaje nic do diagnozy), jednak powoduje wzrost ilości rozproszonego promieniowania, co przekłada się na niższy kontrast zdjęcia)
    \item stosowanie kratki przeciwrozproszeniowej - zapobiega przedostawaniu się promieniowania rozproszonego
    \item odległość ognisko lampy-badany obiekt-rejestrator obrazu - wpływa na ostrość obrazu, małe ognisko oraz niewielka odległość badanego obiektu od rejestratora daje lepszą ostrość
\end{itemize}

Rodzaje filtrów:
\begin{itemize}
    \item Filtry rentgenowskie - zmiana widma promieniowania przez zastosowanie ośrodka pochłaniającego
    \item Filtr własny - bańska szklana, okno kołpaka, olej transformatorowy
    \item Filtry dodatkowe - mocowany na zewnątrz kołpaka, może być absorpcyjny, charakterystyczny lub kompensacyjny
    \item Filtracja całkowita - suma filtracji własnej i dodatkowej 
\end{itemize}

Kratki przeciwrozproszeniowe (współczynnik wypełnienia, liczba $linii/cm$, gęstość powierzchniowa ołowiu w kratce $g/cm^2$, efektywność kratki, absorpcja, jakość kratki)

3 główne parametry obrazowania:
\begin{itemize}
    \item napięcie lampy - różnica potencjałów przyłożonych do anody i katody lampy rentgenowskiej. Zwykle napięcie lampy rentgenowskiej jest wyrażone przez wartość szczytową w kilowoltach ($kV$). Im wyższa wartość napięcia, tym krótsza fala promieniowania, wyższa energia i przenikliwość, a co za tym idzie wyższe „zaczernienie” obrazu;
    \item prąd lampy - prąd elektryczny wiązki elektronów padających na tarczę lampy rentgenowskiej. Zwykle prąd lampy rentgenowskiej jest wyrażony wartością średnią w miliamperach (mA). Prąd katody determinuje jej temperaturę, im wyższa temperatura, tym większa ilość emitowanych elektronów, a co z tym związane – większa ilość kwantów promieniowania
    \item czas ekspozycji - czas trwania napromieniania, zdefiniowany zależnie od określonej metody, zwykle czas, w którym moc wielkości promieniowania przekracza określony poziom. Im dłuższy czas ekspozycji, tym większa dawka, a więc i „zaczernienie” obrazu.
\end{itemize}

Tryby pracy aparatu RTG:
\begin{itemize}
    \item technika dwupunktowa (parametrami są napięcie lampy [kV] oraz obciążenie prądowo-czasowe [mAs])
    \item technika trzypunktowa (3 główne parametry obrazowania)
\end{itemize}

Parametry obrazu rentgenowskiego:
\begin{itemize}
    \item rozdzielczość obrazu
    \item kontrast obrazu
    \item ostrość obrazu
\end{itemize}

\href{https://brain.fuw.edu.pl/edu/index.php/Obrazowanie:Obrazowanie_Medyczne/Metody_obrazowania_medycznego_wykorzystuj%C4%85ce_promieniowanie_rentgenowskie}{Link do strony Brain Wiki: "Obrazowanie Medyczne$/$Metody obrazowania medycznego wykorzystujące promieniowanie rentgenowskie}

\section{Wykład (16.10.2023)}

Układ AEC

Układ IBS - fluoroskopia

HU - jednostka obciążenia cieplnego anody (ma wymiar energii)

Dla generatora jednopulsowego
$\mathrm{HU} = 1 kV \cdot mA \cdot s$

Sześciopulsowego
$\mathrm{HU} = 1,35 kV \cdot mA \cdot s$

Dwunastopulsowego
$\mathrm{HU} = 1,41 kV \cdot mA \cdot s$

\subsection{Mammografia}

W niektórych przypadkach istnieje szczególna konieczność zmniejszenia rozproszeń.

Badanie radiograficzne tkanki miękkiej

Wartości gęstości i efektywnego Z dla wybranych tkanek ludzkich

Tkanka gruczołowa, tłuszczowa i włóknista

Linear attenutaion coefficient (problem odróżnienia tkanki włóknistej i gruczołowej - przy niskich energiach promieniowania różnica jest możliwa do zaobserwowania (ok. 20 keV), efekt fotoelektryczny)

X oddziałuje z tkanką w skutek rozproszenia komptonowskiego i efektu fotoelektrycznego

Konwencjonalne lampy RTG emitują promieniowanie o energiach 70-100 keV.

Energia w zakresie 18-23 keV w zależności od grubości i składu piersi.

Wykres optymalnych energii dla mammografii.

Wykorzystanie promieniowania charakterystycznego (widmo z anody wolframowej filtrowane molbdenem lub Rh)

Zestawienia anoda/filtracja
\begin{itemize}
    \item Mo-Mo
    \item Mo-Rh
    \item Rh-Rh
    \item Wolfram-Rh 
\end{itemize}

Molibden Z = 42, Wolfram Z = 74, przez co w anodzie molibdenowej dominuje promieniowanie charakterystyczne

Rh (Z = 45)

Zakres napięć anody: 24-28 keV. Jeśli napięcie lampy jest zbyt niskie, wartość mAs może rosnąć do nieakceptowalnych wartości, zwiększając niebezpiecznie dawkę promieniowania.

Mammograf - budowa

Płytka kompresująca (zmniejsza się grubość sutka, a więc zmniejsza się rozporoszenie promieniowania i dzięki temu rośnie rozdzielczość obrazu)

Wiązka promieniowania musi być odpowiednio ułożona (kolimacja wiązki) - system jest tak zubodowany, że pionowa wiązka przechodzi równolegle do ciała pacjentki.

Rozmiar ogniska jest bardzo istotnym parametrem lampy mammograficznej.

System DIcom

Rozdzielczość paru linii\slash cm.

Mammografy mają dwie wielkości ogniska: 0.3 i 0.1 mm

ACR accredition phantom

Tomosynteza

Aparaty stomatologiczne
\begin{itemize}
    \item wewnątrzustne
    \item zewnątrzustne (pantomografia (przygotowania do założenia aparatu na zęby), cefalografia)
    \item aparaty 3D
\end{itemize}

Aparaty z ramieniem C (radiologia zabiegowa, angiografia)

Aparaty densytometryczne

Aparaty przyłóżkowe

Tomografia klasyczna

\subsection{Podstawowe cechy nowotworów}

KRN - krajowy rejestr nowotworów

Nowotwór - grupa chorób, w których komórki organizmu dzielą się w niekontrolowany sposób, a nowo powstałe komórki nowotworowe nie różnicują się w typowe tkanki.

Nowotwór - czyli niekontrolowana proliferacja komórek. Wszystkie komórki nowotworowe są klonami pojedynczej komórki.

Żeby powstał nowotwór złośliwy wymaga to zgromadzenia kilku mutacji w komórce (6-8).

1. Mutacje w genach kontrolujązych mitozę
2. W genach regulujących proces apoptozy (zaprogramowanej śmierci komórki)

4. Stymulujące angiogenezę - tworzenie się nowych naczyń krwionośnych niezbędneych do rozwoju guza nowotworowego
5. Stymulujące powstanie przerzutów - roznosić drogą układu ochronnego i krwionośnego

Cechy charakterystyczne nowotworów:
\begin{itemize}
    \item Szybki wzrost
    \item Naciekanie i niszczenie okolicznych tkanek
    \item Zdolność do przerzutów do węzłów chłonnych i odległych narządów
    \item Zaburzenia apoptozy
    \item Anaplazja
    \item Wznowy miejscowe
    \item brak torebki guza
    \item odrastanie w miejscu występowania po niedokładnym usunięciu pierwszej zmiany
    \item duża zdolność tworzenia nowych kompensacyjny
    \item duże zróżnicowanie wyglądu komórek nowotworowych
\end{itemize}

Cykl komórkowy - fazy
\begin{itemize}
    \item Mitoza (6-8)
    \item wzrost G1 (1-8)
    \item synteza DNA S - replikacja (6-8 h)
    \item wzrost G2 (2-4)
\end{itemize}

Cykl komórkowy w komórkach nowotworowych jest krótszy niż w zdrowych.

Proliferacja - zdolność rozmnażania się komórek; jedna z cech organizmów żywych

Kontrola cyklu komórkowego

Częstość wchodzenia w fazy S i M cyklu jest różna w zależności od rodzaju komórki.

Nadmierna apoptoza jest hamowana przez "geny przeżycia".

Onkogeneza (przez 5 lat pacjent uznawany jest jako pacjent onkologiczny)

Radiowrażliwość tkanki nowotworowej i prawidłowej - komórki wykazują różną wrażliwość na promieniowanie jonizujące. Wykresy prawd-stwa kontroli nowotworu

Stosując radioterapie dązy się do jak największego TCP i jak najmniejszego NTCP (zazwyczaj 0,05).

Radiochirurgia

Terapia radioizotopowa

Terapia radykalna - prowadzona z zamiarem wyleczenia choroby nowotworowej, i paliatywna - poprawa jakości życia w okresie kiedy choroby nie można zatrzymać

Nowotwór to nieprawidłowa tkanka powstająca z jednej "chorej" komórki organizmu. Nowotwory mogą być łagodne i złośliwe.

Rozwój nowotworu:
\begin{itemize}
    \item Zmiana przednowotworowa - wiąże się z rysykiem rozwoju nowotworu złośliwego
    \item Stan przednowotworowy - choroba związana ze zwiększonym ryzykiem wystąpienia nowotworu złośliwego
\end{itemize}

Nowotwory złośliwe:
\begin{itemize}
    \item pochodzenia nabłonkowego (raki)
    \item z komórek mezenchymalnych (mięsaki)
    \item tkanka limfatyczna i ukł. krwionośnego (chłoniaki, białaczki)
\end{itemize}
Występują również nowotwory pochodzące z piwerwotnej komórki płciowej.

Nowotwory łagodne - cechy
ostre odgraniczenie guza
rozprężający typ wzrostu
brak zdolności tworzenia przerzutów

Wyjątek stanowią naczyniaki - nowotwór łagodne nie będące otoczone torebką i wciskają się nieregularnymi wypustkami pomiędzy komórki narządu.

\end{document}