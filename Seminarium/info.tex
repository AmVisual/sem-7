\documentclass{article}
\usepackage{polski}
\title{Metody i Techniki Jądrowe - wykąłdy}
\author{Maciej Standerski}
\begin{document}
\maketitle

\section{09.10.2023}


\subsection{Wstęp}

Referat: 20.11

Należy pokazać wiedzę ze swojej dziedziny

W referacie należy zawrzeć, jaki jest stan wiedzy na dany temat. Przedstawić, co jest naszą pracą, a co jest pracą innych.

Należy wprowadzić w temat.

Elementy teorii.

Należy przedstawić, z czego skorzystaliśmy.

\subsection{Przysposobienie biblioteczne z elementami informacji naukowej}

Źródła informacji naukowej

Wyszukiwanie informacji, startegia wyszukiwania

Gdzie szukać:
- katalogi bilb.
- bazy danych
- wyszukiwarki prac

Układ działowy zbiorów

Informcaje o zasobach bibl. naukowych i akademickich.

Katalog NUCAT, w świecie OCLC

Baza danych:
- artykuły, ksiązki el.
- dysertacje i rozprawy naukowej
- materiały konferencyjne
- opisy patentowe
- dane statystyczne

Biblioteki cyfrowe
- zasoby, których prawa autorskie wygasły
- biblioteka cyfrowa PW (kolekcje)

Wyszukiwarki naukowe

FreeFullPDF

Google scholar
Google books
FreeFullPDF

Sformułowanie słów kluczowych i wyrażeń oddających przedmiot problemu

Sformułowanie odpowidzi do tych słów synonimów

Określenie baz i serwisów, które zostaną poddane wyszukiwaniu

Rozpoczęcie wyszukiwania

Zalecana kolejność przeszukiwania zasobów bibliotecznych:

Multiwyszukiwarka

Bibl. PW Katalog

Wyszukiwanie zaawansowane

Lista e-baz (na stronie Bibl. PW pod wyszukiwaniem)

IbukLibra

Springer Link

Bazy pełnotekstowe:
AIP, APS, DOAJ, IbukLibra, IOPSCIENCE, ProQuestEbookCentral

Scopus

Baza wiedzy PW

Jeśli w bibliotece nie ma dostępu do danych prac, można skorzystać z:

- wypożyczalni międzybibliotecznej

- BiblioWawy (konto należy założyć w bibliotece PW, później można korzystać z tej biblioteki)

Zasady sporządzania opisów bibliograficznych

\end{document}