\documentclass{article}
\usepackage{polski}
\usepackage{hyperref}
\usepackage[margin=2.5cm]{geometry}

\setlength{\parindent}{0pt}

\title{Wykłady z Podstaw technik obrazowania w medycynie}
\author{Maciej Standerski}
\begin{document}
\maketitle

\section{Zadanie 1}

\subsection{Rozdzielczość detektora RTG}

\subsection{Wyznaczyć odległość detektora od lampy}

\subsection{Rozmiar ogniska lampy RTG}
10 - 90 \%

\section{Zadanie 2}

Przy napięciach 50 i 54 kV

Warstwa połowicznego osłabienia

Wykorzystujemy metodę schodkową

Im wyższa energia promieniowania tym współczynnik osłabienia jest mniejszy. Tym samym HVL rośnie.

Utwardzanie promieniowania X (filtrując promieniowanie (na anodzie, na okienku i filtrze przy lampie) rośnie średnia energia, a zatem maleje współczynnik osłabienia)

Krzywe utwardzają się.

Wnioski:

Im wyższa energia, tym mniejszy współczynnik osłabienia.

Obserwujemy efekt utwardzania wiązki, czyli silniejsze osłabianie niższych energii i słabsze osłabienie wysokich energii.

\section{Zadanie 3}

Płytki aluminiowa i ołowiana

Zależność kontrastu od napięcia

10 zdjęć przy różnych napięciach i obliczenie kontrastu

Startujemy od 40 kV i zmieniamy co 5 kV (wyjątek 76 zamiast 75 kV). 40 - 76 kV

Wraz ze wzrostem napięcia rośnie natężenie

Detektor ma pewien zakres dynamiczny (w pewnym momencie detektor nie rejestrował już wzrostu sygnału)

Im lżejszy pierwiastek, tym optymalne napięcie ze względu na kontrast będzie mniejszy.

Dla niskich - energie są zbyt niskie, co wpływa na niski kontrast

Wysokie energie - wysycenie się tła, detektor nie rozpoznaje już, bo jest poza zakresem detektora.

\section{Zadanie 4}

Pomiar MTF

Pomiar rozdzielczości ołowianej płytki

\end{document}