\documentclass{article}
\usepackage{polski}
\usepackage{hyperref}
\usepackage[margin=2.5cm]{geometry}

\setlength{\parindent}{0pt}

\title{Wykłady z Podstaw technik obrazowania w medycynie}
\author{Maciej Standerski}
\begin{document}
\maketitle

\section{Wykład (04.10.2023)}
\subsection{Przebieg i regulamin przedmiotu}

\textbf{Prowadzący}: dr inż Piotr Brzeski

\textbf{Konsultacje}: p. 60, 422 (IRiTM), poniedziałki, wtorki (rzadko), środy, czwartki (rzadko)

\textbf{Materiały online}: \href{https://studia.elka.pw.edu.pl/pl/23Z/-/login/}{https:\slash \slash studia.elka.pw.edu.pl\slash pl\slash 23Z\slash-\slash login\slash}

\textbf{e-mail}: piotr.brzeski@pw.edu.pl

\textbf{Laboratoria}: zaczynają sie pod koniec października

\textbf{Zaliczenie}:

Wykład:

Egzamin 0 i w sesji, 3 kolokwia (2h), średnia ocena z kolokwiów lub z egzaminu

Laboratoria:

1 laboratorium może być niezaliczone. W ocenę z laboratoriów wliczane są punkty ze wszystkich labów, również z niezaliczonych. Na ostatnich laboratoriach jest zaliczenie. Zajęcia laboratoryjne oceniane są na podstawie wejściówki oraz sprawozdania

Ocena końcowa $ = 0.5 O_W + 0.45 O_L + 0.05 O_{ZL}$, gdzie $O_W$ - ocena z wykładu, $O_L$ - ocena z laboratorium, $O_ZL$ - ocena z zaliczenia laboratorium

\textbf{Serwer}: Lkstudia3

\textbf{Literatura}: Wykłady z obrazowania medycznego na Politechnice Gdańskiej

\textbf{Harmonogram wykładu}:
\begin{itemize}
    \item 4 wykłady
    \item kolokwium 1
    \item 3 wykłady
    \item kolokwium 2
    \item 3 wykłady
    \item kolokwium 3
    \item egzamin
\end{itemize}


\subsection{Czym jest obrazowanie?}

Przedstawienie pewnej cechy fizycznej organizmu w postaci obrazu, zwykle 2D

(rozkład radiofarmaceutyku w tkankach – gamma kamera, scynytygrafia lub gammagrafia)

(gęstość protonów w tkankach – tomografia rezonansu magnetycznego)
Rzutowanie obiekt 3D (+ ewentualnie czas) na obraz 2D

\subsection{Cel obrazowania medycznego}
\begin{itemize}
    \item wgląd w anatomię struktur wewn organizmu i ich fizjologie
    \item analiza i interpretacja obrazów w diagnostyce
\end{itemize}

\subsection{Systemy obrazowania medycznego}

\begin{itemize}
    \item RTG,
    \item tomografia komputerowa,
    \item scyntygrafia, 
    \item tomografia PET, 
    \item tomografia rezonansu magnetycznego, 
    \item USG, 
    \item termografia,
    \item SPECT (tomografia pojedynczego fotonu)
\end{itemize}

\subsection{Od czego zależy wartość diagnostyczna?}

\begin{itemize}
    \item jakość obrazu (kontrast, rozdzielczość przestrzenna, stosunek sygnału użytecznego do szumu SNR, poziom artefaktów, poziom zniekształceń przestrzennych),
    \item warunki obserwacji,
    \item wiarygodność diagnostyczna,
    \item charakterystyka pracy lekarza-specjalisty
\end{itemize}

Rozdzielczość obrazowania - najmniejsza odległość w obiekcie obrazowanym między dwoma punktami o maksymalnym kontraście które można rozróżnić jako dwa obiekty na obrazie (FWHM – Full Width at Half Maximum, FWTM – Full Width at Tenth Maximum)

Rozdzielczość wartstwy może być większa (np. ok 1 mm)

Jakość obrazu - funkcja przenoszenia modulacji - iloraz między kontrastem fizycznym do kontrastu obrazowego $\mathrm{MTF} = \frac{K_f}{K_o}$ gdzie $K_f$ jest kontrastem fizycznego obiektu, natomiast $K_o$ to kontrast danego obiektu na obrazie.

Więcej o MTF i CTF (fukncji przenoszenia kontrastu) na stronie pod \href{https://brain.fuw.edu.pl/edu/index.php/Obrazowanie:Obrazowanie_Medyczne/Podstawowe_Parametry_Obraz%C3%B3w#label-fig:modulation_depth}{linkiem}

Wzorce w postaci sinusoidy.

$MTF = 1$ - gdy częstotliwość danego wzorca wynosi zero, obraz jest idealnie odwzorowany. Wraz z częstotliwością maleje funkcja przenoszenia modulacji.

Funkcja przenoszenia modulacji całkowitej jest iloczynem funkcji przenoszenia modulacji składowych systemu obrazowania.

$MTF_c = \displaystyle\prod_{i=1}^{N}\mathrm{MTF}_i$, gdzie $N$ - jest liczba systemów.

Na kolokwium

Stosunek sygnał/szum (NSR)

Artefakty

Zniekształcenia:
\begin{itemize}
    \item geometryczne
    \item zmiany pola magnetycznego
    \item przekrzywienie powierzchni odbiornika
    \item spowodowane fizjologią tkanek, narządów
\end{itemize}

\subsection{Warunki obserwacji}

Negatoskop - silna lampa do podświetlania negatywów z RTG

Wiarygodność diagnostyczna obrazów

Jeśli przesuniemy próg decyzyjny, może sprawić, że diagnoza będzie błędna.

Procedura decyzyjna

Macierz decyzyjna (prawdziwie pozytywna TP, fałszywie pozytywna FP, fałszywie negatywna FN, prawdziwie negatywna TN)

Wiarygodność metody diagnostycznej

Na kolokwium

$\mathrm{czulosc} = TP/(TP+FN)$

$\mathrm{specyficznosc} = TN/(TN+FP)$

$\mathrm{dokladnosc} = (TP + TN)/(TP + TN + FP + FN)$

$\mathrm{pozytywna~wartosc~predykcyjna} = TP/(TP + FP)$

$\mathrm{negatywna~wartosc~predykcyjna} = (TN)/(TN + FN)$

Krzywa ROC - służy do oceny metody badawczej

Krzywa A - czysto przypadkowa (pratrz wykład 1-2)

\section{Wykład (11.10.2023r.)}

\subsection{Techniki Medycyny Nuklearnej}

Funkcje:
\begin{itemize}
    \item terapia
    \item diagnostyka
\end{itemize}

Jak wprowadzić izotop promieniotwórczy do ciała:

- doustnie
- dożylnie
- przez układ oddechowy

Radiofarmaceutyk:
Podział w zależności od sposobu uzyskiwania:
- radionuklidy reaktorowe - powstające w wyniku reakcji jądrowej danego izotopu i neutronów
- radionuklidy cyklotronowe 
- radionuklidy generatorowe

Technet-99m (Tc-99m) generatorowy - najczęściej stosowany izotop promieniotwórczy stosowany w radiodiagnostyce.

Gammakamera - w scyntygrafii to urządzenie diagnostyczne do badań narządów, w których nagromadzony jest radioizotop. Wyposażony jest on w detektor o dużym polu widzenia. W detektorze znajduje się kryształ scyntylacyjny, który pod wpływem promieniowania jonizującego (najczęściej gamma) emituje błyski świetlne (scyntylacje). Źródło Wikipedia

Jakie własności musi mieć dobry izotop promieniotwórczy
\begin{itemize}
    \item fizyczne \begin{itemize}
        \item $T_{1/2}$ - czas połowicznego rozpadu - musi być na tyle krótki, aby nie przekroczyć maksymalnej dawki dla pacjenta, oraz na tyle długi, aby móc go przetransporotwać. Dla przykładu czas połowicznego rozpadu dla technetu-99m wynosi 6h.
        \item $A~[Bq]$ - jednostką w układzie SI jest Bekerel [$1 Bq = 1\mathrm{rozpad} / 1s$], dawniej jednostką aktywności były kiury [$1 Ci = 3,7 \cdot 10^{10} Bq$]. Aktywność promieniotwórcza musi być na tyle duża, aby promieniowanie mogło opóścić organizm pacjenta, ale też nie za duża, żeby nie przekroczyć dopuszczalnej dawki
        \item Rozpad gamma monoenergetyczny - najbardziej przenikliwy rodzaj promieniowania, dzięki temu możliwe jest obrazowanie na podstawie promieniowania wydostającego się z ciała pacjenta.
        \item Energia - wykorzystywane są izotopy o energiach $30 - 360~keV$, najczęściej stosowany izotop - technet posiada dla przykładu energię $141~keV$
    \end{itemize}
    \item Chemiczne \begin{itemize}
        \item Czystość - ile izotopu promieniotwórczego zawiera dana próbka. Czysty izotop to taki, który zawiera dużą ilość tego izotopu, bez znaczącej ilości innych pierwiastków (zanieczyszczeń). 
    \end{itemize}
    \item Biologiczne \begin{itemize}
        \item Powinowactwo do badanego organu\slash narządu - zdolność substancji do wiązania się z komórkami określonego narządu\slash tkanki\slash organu. Dzięki temu, że radiofarmaceutyki wiążą się z konkretnymi organami, gromadzą się one w nich i dzięki ich dużej koncentracji w określonym organie pozwala na analizę jego struktury i fizjologi. Różne radiofarmaceutyki mogą wiązać się z różnymi organami. Np. jod gromadzi się w tarczycy.
        \item Czas rozpadu biologicznego $T_B$ - jest to czas wydalania radioizotopu z organizmu. Pożądany czas rozpadu biologicznego musi być wystarczająco długi, aby móc wykonać badanie diagnostyczne, oraz na tyle krótki, aby pacjent szybko pozbył się farmaceutyku z organizmu po badaniu.
    \end{itemize}
\end{itemize}

Czas połowicznego rozpadu i czas rozpadu biologicznego razem składają się na tzw. efektywny czas połowicznego zaniku.

\section{Wykład (18.10.2023)}

Rozkład izotropowy

Gammakamera, Scyntykamera, Kamera Angera

Detektor składa się z:
\begin{itemize}
    \item Kolimatora (z ołowiu z dodatkiem antymonu do utwardzenia, w kolimatorze znajdują się otwory które są prostopadłe do jego powierzchni) - przepuszczenie fotonów, które są prostopadłe do powierzchni kolimatora
    \item Scyntylator (kryształ scyntylacyjny NaI aktywowany Tl (aktywacja talem zwiększa oddziaływanie kryształu z promieniowaniem gamma))
    \item Fotopowielacze
    \item Procesor pozycyjny
\end{itemize}

Scyntygraf - nie jest już używana

Rozmiary kryształu: powierzchnia 500mmx400mm

Kwant gamma oddziaływuje z materią na 3 sposoby (na egzaminie dyplomowym):
\begin{itemize}
    \item fotoefekt
    \item efekt komptona
    \item tworzenie par (elektron pozyton) energia kwantu powyżej ok. 1 MeV
\end{itemize}

Obraz scyntygraficzny jest obrazem całkowym (całkowanie po warstwach ciała pacjenta).

Własności kryształu scyntylayjnego:
- higroskopijny

Gammakamery: jedno, dwu i trójgłowicowe

Fotopowielacze powinny pokryć całą powierzchnię kryształu

Scyntylator po oddziaływaniu z kwantami gamma wysyła fotoelektrony.

Napięcie w fotopowielaczu: 1 kV.

Elektron powstający na fotokatodzie jest nakierowywany przez pole elektryczne na peirwszą dynodę. Osiąga on energię wystarczającą, aby wybić większą ilość el. na kolejnej dynodzie. W fotopowielaczu znajduje się rząd dynod.
Fotopowielacz wytwarza ok 10 000 000 elektronów.

Fotopowielacze najlepiej aby miały kształt heksagonalny.

Jak z rozkładu kwantów odzyskać położenie (x,y) kwantu gamma.

- Układ sumowania liniowego
- Formowanie i kodowanie sygnałów pozycyjnych w gammakamerach (model Tanaki) (sygnały: STROB - sygnał, który określa, czy dane współrzędne (x,y) są ważne, X, Y - koordynaty błysku świetlnego)
- Do każdego fotopowielacza dać po jednym wzmiacniaczu, i każdy podłączyć do przetwornika a-c

Skąd beirze się STROB - łączymy wszystkie anody fotopowielaczy i wspólnie podajemy na jednokanałowy analizator amplitudy (urządzenie elektroniczne, które na wejściu otrzymuje sygnał, który jest sumą sygnałów z wszystkich fotopowielaczy, i jego amplituda jest proporcjonalna do energii kwantu promieniowania gamma, i na wyjściu sygnał jest podawany tylko wtedy, gdy amplituda sygnału na wejściu znajduje się w dopuszczalnym przedziale wartości amplitudy (jeśli amplituda jest za niska, lub za wysoka, na wyjściu brak sygnału))

Do czego służy jednokanałowy analizator amplitudy:
Do usunięcia: Impulsy komptonowskie, podówjne impulsy lub impulsy, które nie pochodzą z organizmu człowieka.

Jak określane jest położenie impulsu świetlnego ze scyntylatora?

Linia opóźniająca, która powoduje rozdwojenie impulsu na dwa. Różnica dotarcia sygnału po obu końcach linii op. pozwala na określenie położenia impulsu. Time to Amplitude converter (amplituda jest proporcjonalna do różnicy czasu między startem i stopem)

Przetwornik analogowo-cyfrowy - wyjście X, Y są sygnałami analogowymi. Przetwornik zamienia go na sygnał cyfrowy.

Kolimator zapobiega tylko zatrzymywaniu kwantów poruszających się pod kątem. Kwanty gamma, które powstają w wyniku efektu komptona zachodzącego w ciele ludzkim ma energię mniejszą niż kwant gamma pochodzący bezpośrednio ze źródła

Rodzaje kolimatorów z równoległymi otworami:
- wysokoenergetyczne (pow. 141 keV)
- niskoenergetyczne (poniżej 141 keV)

Różnią się wielkością otworów, grubością ścianek

Kolimatory typu pinhole (jego działanie opeira się na działaniu kamery Obscura, naprzykład przy badaniach tarczycy (małych narządów))

Dawniej korzystano z:
Kolimatory dywergentne i konwergentne

cal - ok 2.5 cm
PMT - photo multiplier tube
FWHM

Układy korekcji

Badanie WholeBody

Flood image, Bar-phantom image

Korekcja nieidealności gammakamery

Tworzenie obrazu w komputerze:
Akwizycja statyczna:
Procesor bierze zawartość wszytskich komórek zarezerwowanych dla obrazu. Kiedy otrzymuje STROB, zatrzymuje wyświetlanie. Przechodzi do obsługi przerwania. Zczytuje X i Y i tworzy adres komórki pamięci ($A = 256Y + X + B_0$). $B_0$ - komórka bazowa, od której zaczyna się zarezerwowana pamięć na obraz. Pod tym adresem dodaje 1. Naprzykład przy badaniach tarczycy.

Akwizycja dynamiczna:
1. Najpierw robimy akwizycję statyczną pierwszego obrazu, następnie przesuwamy się w pamięci o obszar jednego obrazu i dokonujmy akwizycji statycznej, itd. Z góry zakładamy czas akwizycji każdego obrazu.

2. List mode pozwala na zapamiętanie wszystkich zdarzeń i uszeregowania ich w czasie. Zapamiętujemy X i Y które zaszły w określonym czasie (np. 10 ms). (255,255) jest znacznikiem oddzielającym kolejne chwile zbierania danych.

Obrazy parametryczne - do określenia parametrów badanego obszaru

Np. wyznaczanie regionu zainteresowania - zbieranie natężenia z wybraego obszaru, czas osiągnięcia maksimum.

3. Akwizycja bramokowana - synchronizowana zewnętrznym impulsem (badania angiokardiografii izotopowej). Na początku wyznaczamy ilość impulsów na sekundę. Obrazy z poszczególnych przedziałów czasu sumujemy, uzyskując obraz pracy serca.

Na kolokwium:
Przy badaniu statycznym (losowy charakter rozpadu i akwizycji)
Czy guz jest gorący czy zmiany

$n_0 - l.impulsow/sek/cm^2$ - obszar normalny

$n_1 - l.impulsow/sek/cm^2$ - obszar patologii

$n_1 * \Delta S * \Delta t + k\sqrt{n_1 \Delta S Delta t} < n_0 \Delta S \Delta t$ - delta t to czas akwizycji, $k=3$ p. 0,99, $k=2$ p. 0,91, k=1 . 0,73 - prawdopodobieństwo wykrycia guza zimnego

Dla guza gorącego w powyższym znaku zmieniamy "+" na "-" i zmienamy kierunek nierówności.

\section{Wykład (25.10.2023)}

\subsection{Ultrasonografia}

Metoda obrazowania narządów wewn. przy pomocy fal akustycznych - ultradźwiękowych.

Głowica z piezoelektrykiem.

Wykorzystywane zjawisko odbicia od struktur różniących się imepdancją akustyczną, a która zależy od gęstości ośrodka.

Powracająca fala nazywana jest echem.

Zjawisko Dopplera

Ultrasonografia Dopplerowska

Niska rozdzielczość (lepsza od scyntygrafii, ale gorsza od radiologii)

Przetworniki, które pełnią rolę nadajnika i odpbiornika sygnałów akustycznych.

W tkankach miękkich i wodzie fale podłużne

W tkankach twardych fale poprzeczne.

Podstawowe parametry fali:
\begin{itemize}
    \item częstotliwość
    \item prędkość
    \item dł. fali
    \item amplituda
    \item natężenie
\end{itemize}

częstotliwość

0 - 20 Hz - infradźwięki

20 Hz - 20 kHz - dźwięki słyszalne

pow. 20 kHz - ultradźwięki

50kHz - 600 kHz - badania kości
200 kHz - 5 MHz - badania przepływów
2 MHz - 10 MHz - obrazowanie tkanek wewn. - najczęściej stosowane

Prędkośc rozchodzenia się ultradźwięków - zależy od właściwości ośrodkach

$v \sim E / \rho $

$E [N/m^2]$ - sprężystość (modół Younga)

$\rho$ - gęstość

Wartości prędkości są zbliżone

$s = v * t$

Długość fali $\lambda = v / f$

Amplituda fali

Konkretnie interesuje nas stosunek amplituda fali padającej do odbitej (amplituda względna) (amplituda powracająca mierzona w decybelach).

$A_rel = 20 \log{A/A_0}$

Oddziaływanie ultradźwięków z materią

Osłabienie w [dB]

Tłumienie T

$T = \alpha * f * x$ - alfa współczynnik tłumienia (tłumienie zależy wykladniczo od częstotliwości, bo tłumienie jest w dB)

Odbicie $R = I_R / I_0 = (z_1 - z_2)^2 / (z_1 + z_2)^2$ ($z = \rho * v$ - oporność akustyczna zależy od sprężystości materiału)

Strata amplitudy podczas odbicia

TGC (time gain control) - wzmacniacz, którego wzmocnienie zależy od czasu.

Gen -> Sonic -> fala ultradźwiękowa -> Sonic -> TGC -> Display

Kolokwium (narysowanie zależności amplitudy fali dźwiękowej od głębokości tkanki (rozchodzenie się fali w tkance)) 

4 rodzaje obrazów USG

- typu A (amplituda) (wzmacniecz - wzmocnienie narasta w czasie

- typu B (brightness) - najczęściej stosowany (kątowy przetwornik)

- typu M (motion) - otrzymujemy obraz struktury w ruchu, propagacja fali w jednym kierunku

- typu D (Doppler) - zaburzenia w przepływie krwi w naczyniach krwionośnych

Głowica USG

Kształtowanie fali ultradźwiękowego. Fale z kazdego kryształu interferują ze sobą tworząc falę wynikową skupioną na określonym przez nas obszarze.

Rodzaje głowic:
\begin{itemize}
    \item liniowa
    \item konweksowa
    \item trapezoidalna
    \item sektorowa (wysyła fale w pewnym sektorze) - wady: 1. powtarzające się obszary przy przesuwaniu, 2. pojawia się obszar martwy
    \item pierścieniowa
    \item wewnątrznaczyniowa
    \item intro-waginalna i intro-rektalna
\end{itemize}

\section{Wykład 15.11.2023}

Radiografia

Kolejne kamienie milowe w hisotrii radiografii

Koronarografia (obrazowanie naczyń w sercu)

Radiografia analogowa

Kontrast w radiografii zależy od trzech procesów:
\begin{itemize}
    \item przenikanie promieniowania (zależy od parametrów promieniwowania X [parametry wpływające na promieniowanie X - napięcie anodowe, ognisko lampy rentgenowskiej (z wolframu wzogacanego renem, a w mammografii z molibdenu)], filtracja, oraz od obiektu, d - grubość, $\rho$ - gęstość obiektu, Z - liczba atomowa)
    \item rozpraszanie - pogarsza kontrast (co pogarsza: zwiększenie rozpraszania - grubość obiektu obrazowanego, zwiększenie pole naświetlane, za duże napięcie anodowe (powoduje wzrost oddziaływań komptonowskich), jak ograniczyć - siatka przeciwrozproszeniowa (Buck'y) $r = t/d$, t - wysokość siatki, d - szerokość pojedynczego otworu siatki r(5,16))
    \item rejestracja - zależy od ekspozycji (liczby kwantów które padają na jednostkę powierzchni, detektor)
\end{itemize}

Filtracja - wycinanie z promieniowania rentgenowskiego kwanty o niskich energiach

Kolimacja

Transmisja i absorpcja

Tworzenie obrazu (przecodzi przez kratkę Bucky'ego)

Wizualizacja obrazu

Najlepiej, aby detektor wykrywał jak największą liczbę kwantów.

W wyniku oddziaływania fotoelektrycznego foton odziałuje z elektronami z powłoki K, czasami z powłoki L

Oddziaływanie komptonowskie - Wybijany elektron komptonowski z zewnętznej powłoki, zmiana kierunku kwantu gamma i zmniejszenie jego energii

Zmiana natężenia promieniowania: prawo Beera $I = I_0 e^{-\mu x}$

Zastosowanie (tym bardziej promieniowanie rozproszone zostanie pochłonięte, im większy jest stosunek $t/d$)

Aby pozbyć się obrazu kratki, należy poruszać kratką.

Zadanie na kolokwium: obliczyć prędkość elektornu, który uderza w anodę.

$mv^2/2 = E_k$, $E_p = eU_a$, $m = m_0 / \sqrt{1-v^2/c^2}$

\section{Wykład 22.11.2023}

Współczynnik absorpcji

Widmo lampy rentgenowskiej - zależność liczby kwantów od energii kwantów

Lampa działa w zakresie energii 100-150 keV

Okienko z berylu (o małej liczbie atomowej)

Skąd bierze się linia prosta 

Gdy elektrony uderzają w ognisko

Promieniowanie charakterystyczne lampy rtg nie zależy od napięcia anody oraz prądu lampy.

Odprowadzanie ciepła z anody lampy rentgenowskiej: lampy rtg z wirującą anodą (tarcza obracająca sprawia, że elektrony nie padają tylko w jedno miejsce anody, grafit i miedź)

Promieniowanie char. w mammografii stanowi 50\% całego promieniowania emitowanego przez lampę rtg.

Mechanizmy generacji promieniowania X: Brehmstrallung - promieniowanie hamowania, promieniowanie charakterystyczne

Zależności geometryczne: wymiary ogniska
większe ognisko sprawia, że powstają większe półcienie, najlepiej aby ognisko było punktowe

Współczesna radiografia analowgowa: detektor obrazu (błona światłoczuła = warstwa poliestru + naniesionena niego krzyształy bromku srebra)

Kaseta - okładka wzmacniająca (scyntylator), podłoże z poliestru, emulsja z bromku srebra (bromki, chlorki metali ziem rzadkich (gadolin, iterb, europ)). 

Okładki pograszają rozdzielczość obrazu w stopniu zależnym od grubości okładek. Przy grubszych okładkach o większej wydajności dają bardziej rozmyty obraz. Rozwiązanie problemu - struktura igłowa (kolumnowa)

Dlaczego stosowana jest radiografia cyfrowa?
- w medycynie coraz powszechniejsze są modalności cyfrowe (obraz powstaje w komputerze)
- rozwój techniki szybkiej transmisji danych w telemedycynie
- rozwój klinicznych sieci typu RIS, HIS, PACS
- rozówj techniki przetwarzania i analizy obrazu, i interpretacji automatycznej, a także wspomagania jednostki
- narasta problem archiwizacji milionów klisz rentgenowskich
- ochrona środowiska - stosowane chemikalia w radiografii są agresywne
- wzrost dostępności usług medycznych coraz trudniej pogodzić z długim czasem obróbki zdjęć rentgenowskich
- dyrektywy UE

Wady i ograniczenia radiografii analogowej
- tworzony jest pojedynczy obrazu
- dostęp do wyników badania ograniczony
- nietrwały nośnik obrazu
- długi czas oczekiwania na wynik badania
- 

Zalety radiografii cyfrowej
- krótki czas badania
- niższe dawki pochłonięte
- scentralizowane gromadzenie i udostepnianie danych
- łatwa archiwizacja i zabezpieczenie danych - powielanie obrazów bez utraty jakości
- elastyzny sposób wizualizacji badań
- możliwości przetwarzania obrazów
- korzystanie  osiągnieć teleradiologii
- wygoda pracy radiologów
- standaryzacja procedur

Jaką wybrać dorege cyfryzacji
- skanowanie konwencjonalnych zdjęć rtg
- radiografia komputerowa CR
- systemy wykorzystujące matryce CCD
- płaskie panele detekcyjne FPD
- inne w tym będące w fazie rozwoju...

Obraz cyfrowy

Liczba poziomów szarości

Archiwizacja obrazu cyfrowego

w medyceynie nie korzystamy z kompresji stratnej

Typowy obraz medyczny wymaga zapisu 10-12 bitowego

Rozmiar: 1 GB dysku można zapisać ok 35 zdjęć

transmisja obrazu cyfrowego

Postprocessing

Ewolucja radiografii cyfrowej

Analogowa - radiografia komputerowa - ccd z ekranem scyntylacyjnym

Digitalizacja błon rtg

Płyty fosforowe BaFBr

Technika odczytu płyt pamięciowych

Charakterystyka putyczna płyty CR (zadymienie błony, dla płyty czułość jest proporcjonalna do energii)

Matryce CCD (wykorzystywane m.in. w aparatach fotograficznych)

Płaskie panele detekcyjne: Thin Film Transistor TFT - wykorzystywane w monitorach

Detetory radiograficzne: FPD - z odczytem pośrednim (światło generuje ładunek) i bezpośrednim (bezpośrendio ładunek jest generowany), stosowany jest jodek cezu

\section{Wykład (29.11.2023)}

Ewolucja radiografii cyfrowej

Techniki obrazowe:
\begin{itemize}
    \item Fluoroskopia - ekrany które się tworzy mają małą luminancja (1 nit - jednostka luminancji, 1 lumen - jednostka strumienia świetlnego $3,8 10^{15} fotonow/s$ fali o długości 540 nm). Jasność (1 cd) strumień złapany w jednym kącie steradiana (1 lumen\slash srd). 1 cd\slash $m^2$ jest luminancją. Typowa wielkość luminancji zależy od tego, jaka dawka promieniowania rtg pada na ekran w czasie (1mR/sek odpowiada 0,14 nit). Promieniowanie czasem mogło przechodzić przez ekran i padało na oko diagnosty. Wzmacniacz obrazu. Wymagania stawiane w radiografii cyfrowej
    \item Angiografia - technika używana do wykrywania patologii w układzie krwionośnym. $I_1 = I_0 e^{-\mu_1 d}$
    \item Tomografia SPECT (Single Photon Emission Computerised Tomography) i PET (Positon Emmsion Tomography). Składanie obrazu z wielu stron (płaszczyzn - obrazów całkowych). W SPECT: Tyle ile jest rzędów w kolimatorze, tyle jest orazów w tomografii. Obrazujemy rozkład izotopu we wnętrzu człowieka. Trzeba zapewnić obrót kamery wokół pacjenta (gammakamera umieszczona na ruchomym obrotowym ramieniu). Czasem stosuje się dwie kamery aby zwiększyć czułość tomografu. Kłopoty z odtworzeniem rozkładu (promieniowanie wychodzące z wnętrza pacjenta jest osłabiane i pewne techniki uwzględnania tego osłabienia należ stosować, aby zapobiec niewłaściwemu odtworzeniu obrazu). Tomografia PET - wprowadzanie izotopów, które rozpadają się z emisją pozytonu (beta plus). Zalety - najczulsza metoda ze wszystkich metod obrazowych, wygodna - dawki nie są duże (nie potrzebny jest kolimator, wykorzystuje się zjawisko koincydencji - emisja kwantów gamma o energiach 511 keV każdy w miejscu anihilacji proton-elektron). Pierścień detektorów - łapanie koincydencji dwóch kwantów gamma. Rejestracja linii odpowiedzi. Możliwość rejestracji, gdzie nastąpiła rejestracja na podstawie różnic (Jaka powinna być dokładność koincydencji aby z daną dokładnością określić położenie danego miejsca anihilacji?). Izotopy wykorzystywane w PET, szczególnie ważny jest Fluor-18, nim oznacza się np. glukozę, stosowana do badania nieprwaidłowości w mózgu (mózg potrzebuje glukozy).
\end{itemize}

Wzmacniacz obrazu (to nie fotopowielacz!!!) - bańka szklana (gorszy kontrast i rozdzielczość względem kliszy)

Wymagania stawiane w radiografii cyfrowej

\section{Wykład (13.12.2023)}

\subsection{Podstawy obrazowania medyznego - tomografia CT}

Tomografia (thomos i grapheos).

Generacje tomografów (piąta generacja nie weszła do zastosowania, zbyt skomplikowana i droga).

Podstawy fizyczne: $\mu_E^t = -1/d \ln{I/I_0}$
Podstawy fizyczne: $\int_{}^{}\mu(x)dx = -1/d \ln{I/I_0}$

Jednostki: $CTnnumber = 1000 \frac{\mu - \mu_{water,73keV}}{\mu_{water,73keV}}$, dla wody ma wartość 0.

Tomograf pierwszej generacji (lampa rtg, 1 detektor, wązka wiązka skolimowana, dzięki czemu jakość była wysoka. Obrót ramy o 180 stopni.)

Tomograf drugiej generacji (ruch posuwisty i ruch obrotowy)

Tomograf 3 gen (rewolucja, lampa rtg i matryca detektorów rozłożona na łuku, na tyle szeroka matryca i wiązka promieniwania, aby obejmowała całe ciało - likwidacja ruchu posuwistego, przyspieszenie obrazowania)

Tomograf 4 gen (detektory dookoła pacjenta)

Tomograf 5 gene (na obwodzie znajduje się ognisko lampy rtg, sterujemy kierunkiem wiązki) 2-3 prototypy na świecie

Tomografia spiralna (załatwia sprawę kilku wartsw, obrót lapmpy rtg i jednocześnie przesuwane jest łóżko z pacjentem, uzyskujemy wówczas więcej warstw zdjętych). Problemy: jak lampa jest podłączona, że kable się nie plączą XD? \textbf{Kolokwium}. 

PM - fotopowielacz

Komora ksenonowa - detektor

Elementy systemu tomografii: 

suwnica tomografu rtg (CT gantry)

Lampa rtg, kolimatory, filtry energetyczne - ciągła praca wymaga wysokiego chłodzenia. Lampa rtg głównie produkuje ciepło.

Kolimatory: pierwszy kolimator lampy rtg - ustala szerokość wiązki, drugi kolimator - ustala pole naświetlania pacjenta między lampą a pacjentem, 3 kolimator przed detektorem - określona wiązka dociera do detektora (jedna wiązka dociera z jednego kierunku).

Detektory scyntylacyjne

System akwizycji danych: element detektora, przetwornik prąd-napięcie, integrator, multiplekser, przetwornik A\slash C.

Łóżko pacjenta - powinno absorbować jak najmniej promieniowania rtg, wykonany z włókna węglowego

Rekontrukcja obrazu

\textbf{Kolokwium} jakie metody rekonstrukcji są stosowane: metoda projekcji wstecznej, metoda iteracyjna. Na slajach z wykładu opis metod. Metoda iteracji algebraicznych (rekonstrukcja obrazu na podstawie całek \textbf{Zadanie na kolokwium})

Z prawa Lamberta-Beera (całki).

Obecnie jedyna stosowana metoda - algorytm filtrowanej projekcji wstecznej



\end{document}