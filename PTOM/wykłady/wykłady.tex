\documentclass{article}
\usepackage{polski}
\usepackage{hyperref}
\usepackage[margin=2.5cm]{geometry}

\setlength{\parindent}{0pt}

\title{Wykłady z Podstaw technik obrazowania w medycynie}
\author{Maciej Standerski}
\begin{document}
\maketitle

\section{Wykład (04.10.2023)}
\subsection{Przebieg i regulamin przedmiotu}

\textbf{Prowadzący}: dr inż Piotr Brzeski

\textbf{Konsultacje}: p. 60, 422 (IRiTM), poniedziałki, wtorki (rzadko), środy, czwartki (rzadko)

\textbf{Materiały online}: \href{https://studia.elka.pw.edu.pl/pl/23Z/-/login/}{https:\slash \slash studia.elka.pw.edu.pl\slash pl\slash 23Z\slash-\slash login\slash}

\textbf{e-mail}: piotr.brzeski@pw.edu.pl

\textbf{Laboratoria}: zaczynają sie pod koniec października

\textbf{Zaliczenie}:

Wykład:

Egzamin 0 i w sesji, 3 kolokwia (2h), średnia ocena z kolokwiów lub z egzaminu

Laboratoria:

1 laboratorium może być niezaliczone. W ocenę z laboratoriów wliczane są punkty ze wszystkich labów, również z niezaliczonych. Na ostatnich laboratoriach jest zaliczenie. Zajęcia laboratoryjne oceniane są na podstawie wejściówki oraz sprawozdania

Ocena końcowa $ = 0.5 O_W + 0.45 O_L + 0.05 O_{ZL}$, gdzie $O_W$ - ocena z wykładu, $O_L$ - ocena z laboratorium, $O_ZL$ - ocena z zaliczenia laboratorium

\textbf{Serwer}: Lkstudia3

\textbf{Literatura}: Wykłady z obrazowania medycznego na Politechnice Gdańskiej

\textbf{Harmonogram wykładu}:
\begin{itemize}
    \item 4 wykłady
    \item kolokwium 1
    \item 3 wykłady
    \item kolokwium 2
    \item 3 wykłady
    \item kolokwium 3
    \item egzamin
\end{itemize}


\subsection{Czym jest obrazowanie?}

Przedstawienie pewnej cechy fizycznej organizmu w postaci obrazu, zwykle 2D

(rozkład radiofarmaceutyku w tkankach – gamma kamera, scynytygrafia lub gammagrafia)

(gęstość protonów w tkankach – tomografia rezonansu magnetycznego)
Rzutowanie obiekt 3D (+ ewentualnie czas) na obraz 2D

\subsection{Cel obrazowania medycznego}
\begin{itemize}
    \item wgląd w anatomię struktur wewn organizmu i ich fizjologie
    \item analiza i interpretacja obrazów w diagnostyce
\end{itemize}

\subsection{Systemy obrazowania medycznego}

\begin{itemize}
    \item RTG,
    \item tomografia komputerowa,
    \item scyntygrafia, 
    \item tomografia PET, 
    \item tomografia rezonansu magnetycznego, 
    \item USG, 
    \item termografia,
    \item SPECT (tomografia pojedynczego fotonu)
\end{itemize}

\subsection{Od czego zależy wartość diagnostyczna?}

\begin{itemize}
    \item jakość obrazu (kontrast, rozdzielczość przestrzenna, stosunek sygnału użytecznego do szumu SNR, poziom artefaktów, poziom zniekształceń przestrzennych),
    \item warunki obserwacji,
    \item wiarygodność diagnostyczna,
    \item charakterystyka pracy lekarza-specjalisty
\end{itemize}

Rozdzielczość obrazowania - najmniejsza odległość w obiekcie obrazowanym między dwoma punktami o maksymalnym kontraście które można rozróżnić jako dwa obiekty na obrazie (FWHM – Full Width at Half Maximum, FWTM – Full Width at Tenth Maximum)

Rozdzielczość wartstwy może być większa (np. ok 1 mm)

Jakość obrazu - funkcja przenoszenia modulacji - iloraz między kontrastem fizycznym do kontrastu obrazowego $\mathrm{MTF} = \frac{K_f}{K_o}$ gdzie $K_f$ jest kontrastem fizycznego obiektu, natomiast $K_o$ to kontrast danego obiektu na obrazie.

Więcej o MTF i CTF (fukncji przenoszenia kontrastu) na stronie pod \href{https://brain.fuw.edu.pl/edu/index.php/Obrazowanie:Obrazowanie_Medyczne/Podstawowe_Parametry_Obraz%C3%B3w#label-fig:modulation_depth}{linkiem}

Wzorce w postaci sinusoidy.

$MTF = 1$ - gdy częstotliwość danego wzorca wynosi zero, obraz jest idealnie odwzorowany. Wraz z częstotliwością maleje funkcja przenoszenia modulacji.

Funkcja przenoszenia modulacji całkowitej jest iloczynem funkcji przenoszenia modulacji składowych systemu obrazowania.

$MTF_c = \displaystyle\prod_{i=1}^{N}\mathrm{MTF}_i$, gdzie $N$ - jest liczba systemów.

Na kolokwium

Stosunek sygnał/szum (NSR)

Artefakty

Zniekształcenia:
\begin{itemize}
    \item geometryczne
    \item zmiany pola magnetycznego
    \item przekrzywienie powierzchni odbiornika
    \item spowodowane fizjologią tkanek, narządów
\end{itemize}

\subsection{Warunki obserwacji}

Negatoskop - silna lampa do podświetlania negatywów z RTG

Wiarygodność diagnostyczna obrazów

Jeśli przesuniemy próg decyzyjny, może sprawić, że diagnoza będzie błędna.

Procedura decyzyjna

Macierz decyzyjna (prawdziwie pozytywna TP, fałszywie pozytywna FP, fałszywie negatywna FN, prawdziwie negatywna TN)

Wiarygodność metody diagnostycznej

Na kolokwium

$\mathrm{czulosc} = TP/(TP+FN)$

$\mathrm{specyficznosc} = TN/(TN+FP)$

$\mathrm{dokladnosc} = (TP + TN)/(TP + TN + FP + FN)$

$\mathrm{pozytywna~wartosc~predykcyjna} = TP/(TP + FP)$

$\mathrm{negatywna~wartosc~predykcyjna} = (TN)/(TN + FN)$

Krzywa ROC - służy do oceny metody badawczej

Krzywa A - czysto przypadkowa (pratrz wykład 1-2)

\section{Wykład (11.10.2023r.)}

\subsection{Techniki Medycyny Nuklearnej}

Funkcje:
\begin{itemize}
    \item terapia
    \item diagnostyka
\end{itemize}

Jak wprowadzić izotop promieniotwórczy do ciała:

- doustnie
- dożylnie
- przez układ oddechowy

Radiofarmaceutyk:
Podział w zależności od sposobu uzyskiwania:
- radionuklidy reaktorowe - powstające w wyniku reakcji jądrowej danego izotopu i neutronów
- radionuklidy cyklotronowe 
- radionuklidy generatorowe

Technet-99m (Tc-99m) generatorowy - najczęściej stosowany izotop promieniotwórczy stosowany w radiodiagnostyce.

Gammakamera - w scyntygrafii to urządzenie diagnostyczne do badań narządów, w których nagromadzony jest radioizotop. Wyposażony jest on w detektor o dużym polu widzenia. W detektorze znajduje się kryształ scyntylacyjny, który pod wpływem promieniowania jonizującego (najczęściej gamma) emituje błyski świetlne (scyntylacje). Źródło Wikipedia

Jakie własności musi mieć dobry izotop promieniotwórczy
\begin{itemize}
    \item fizyczne \begin{itemize}
        \item $T_{1/2}$ - czas połowicznego rozpadu - musi być na tyle krótki, aby nie przekroczyć maksymalnej dawki dla pacjenta, oraz na tyle długi, aby móc go przetransporotwać. Dla przykładu czas połowicznego rozpadu dla technetu-99m wynosi 6h.
        \item $A~[Bq]$ - jednostką w układzie SI jest Bekerel [$1 Bq = 1\mathrm{rozpad} / 1s$], dawniej jednostką aktywności były kiury [$1 Ci = 3,7 \cdot 10^{10} Bq$]. Aktywność promieniotwórcza musi być na tyle duża, aby promieniowanie mogło opóścić organizm pacjenta, ale też nie za duża, żeby nie przekroczyć dopuszczalnej dawki
        \item Rozpad gamma monoenergetyczny - najbardziej przenikliwy rodzaj promieniowania, dzięki temu możliwe jest obrazowanie na podstawie promieniowania wydostającego się z ciała pacjenta.
        \item Energia - wykorzystywane są izotopy o energiach $30 - 360~keV$, najczęściej stosowany izotop - technet posiada dla przykładu energię $141~keV$
    \end{itemize}
    \item Chemiczne \begin{itemize}
        \item Czystość - ile izotopu promieniotwórczego zawiera dana próbka. Czysty izotop to taki, który zawiera dużą ilość tego izotopu, bez znaczącej ilości innych pierwiastków (zanieczyszczeń). 
    \end{itemize}
    \item Biologiczne \begin{itemize}
        \item Powinowactwo do badanego organu\slash narządu - zdolność substancji do wiązania się z komórkami określonego narządu\slash tkanki\slash organu. Dzięki temu, że radiofarmaceutyki wiążą się z konkretnymi organami, gromadzą się one w nich i dzięki ich dużej koncentracji w określonym organie pozwala na analizę jego struktury i fizjologi. Różne radiofarmaceutyki mogą wiązać się z różnymi organami. Np. jod gromadzi się w tarczycy.
        \item Czas rozpadu biologicznego $T_B$ - jest to czas wydalania radioizotopu z organizmu. Pożądany czas rozpadu biologicznego musi być wystarczająco długi, aby móc wykonać badanie diagnostyczne, oraz na tyle krótki, aby pacjent szybko pozbył się farmaceutyku z organizmu po badaniu.
    \end{itemize}
\end{itemize}

Czas połowicznego rozpadu i czas rozpadu biologicznego razem składają się na tzw. efektywny czas połowicznego zaniku.

\section{Wykład (18.10.2023)}

Rozkład izotropowy

Gammakamera, Scyntykamera, Kamera Angera

Detektor składa się z:
\begin{itemize}
    \item Kolimatora (z ołowiu z dodatkiem antymonu do utwardzenia, w kolimatorze znajdują się otwory które są prostopadłe do jego powierzchni) - przepuszczenie fotonów, które są prostopadłe do powierzchni kolimatora
    \item Scyntylator (kryształ scyntylacyjny NaI aktywowany Tl (aktywacja talem zwiększa oddziaływanie kryształu z promieniowaniem gamma))
    \item Fotopowielacze
    \item Procesor pozycyjny
\end{itemize}

Scyntygraf - nie jest już używana

Rozmiary kryształu: powierzchnia 500mmx400mm

Kwant gamma oddziaływuje z materią na 3 sposoby (na egzaminie dyplomowym):
\begin{itemize}
    \item fotoefekt
    \item efekt komptona
    \item tworzenie par (elektron pozyton) energia kwantu powyżej ok. 1 MeV
\end{itemize}

Obraz scyntygraficzny jest obrazem całkowym (całkowanie po warstwach ciała pacjenta).

Własności kryształu scyntylayjnego:
- higroskopijny

Gammakamery: jedno, dwu i trójgłowicowe

Fotopowielacze powinny pokryć całą powierzchnię kryształu

Scyntylator po oddziaływaniu z kwantami gamma wysyła fotoelektrony.

Napięcie w fotopowielaczu: 1 kV.

Elektron powstający na fotokatodzie jest nakierowywany przez pole elektryczne na peirwszą dynodę. Osiąga on energię wystarczającą, aby wybić większą ilość el. na kolejnej dynodzie. W fotopowielaczu znajduje się rząd dynod.
Fotopowielacz wytwarza ok 10 000 000 elektronów.

Fotopowielacze najlepiej aby miały kształt heksagonalny.

Jak z rozkładu kwantów odzyskać położenie (x,y) kwantu gamma.

- Układ sumowania liniowego
- Formowanie i kodowanie sygnałów pozycyjnych w gammakamerach (model Tanaki) (sygnały: STROB - sygnał, który określa, czy dane współrzędne (x,y) są ważne, X, Y - koordynaty błysku świetlnego)
- Do każdego fotopowielacza dać po jednym wzmiacniaczu, i każdy podłączyć do przetwornika a-c

Skąd beirze się STROB - łączymy wszystkie anody fotopowielaczy i wspólnie podajemy na jednokanałowy analizator amplitudy (urządzenie elektroniczne, które na wejściu otrzymuje sygnał, który jest sumą sygnałów z wszystkich fotopowielaczy, i jego amplituda jest proporcjonalna do energii kwantu promieniowania gamma, i na wyjściu sygnał jest podawany tylko wtedy, gdy amplituda sygnału na wejściu znajduje się w dopuszczalnym przedziale wartości amplitudy (jeśli amplituda jest za niska, lub za wysoka, na wyjściu brak sygnału))

Do czego służy jednokanałowy analizator amplitudy:
Do usunięcia: Impulsy komptonowskie, podówjne impulsy lub impulsy, które nie pochodzą z organizmu człowieka.

Jak określane jest położenie impulsu świetlnego ze scyntylatora?

Linia opóźniająca, która powoduje rozdwojenie impulsu na dwa. Różnica dotarcia sygnału po obu końcach linii op. pozwala na określenie położenia impulsu. Time to Amplitude converter (amplituda jest proporcjonalna do różnicy czasu między startem i stopem)

Przetwornik analogowo-cyfrowy - wyjście X, Y są sygnałami analogowymi. Przetwornik zamienia go na sygnał cyfrowy.

Kolimator zapobiega tylko zatrzymywaniu kwantów poruszających się pod kątem. Kwanty gamma, które powstają w wyniku efektu komptona zachodzącego w ciele ludzkim ma energię mniejszą niż kwant gamma pochodzący bezpośrednio ze źródła

Rodzaje kolimatorów z równoległymi otworami:
- wysokoenergetyczne (pow. 141 keV)
- niskoenergetyczne (poniżej 141 keV)

Różnią się wielkością otworów, grubością ścianek

Kolimatory typu pinhole (jego działanie opeira się na działaniu kamery Obscura, naprzykład przy badaniach tarczycy (małych narządów))

Dawniej korzystano z:
Kolimatory dywergentne i konwergentne

cal - ok 2.5 cm
PMT - photo multiplier tube
FWHM

Układy korekcji

Badanie WholeBody

Flood image, Bar-phantom image

Korekcja nieidealności gammakamery

Tworzenie obrazu w komputerze:
Akwizycja statyczna:
Procesor bierze zawartość wszytskich komórek zarezerwowanych dla obrazu. Kiedy otrzymuje STROB, zatrzymuje wyświetlanie. Przechodzi do obsługi przerwania. Zczytuje X i Y i tworzy adres komórki pamięci ($A = 256Y + X + B_0$). $B_0$ - komórka bazowa, od której zaczyna się zarezerwowana pamięć na obraz. Pod tym adresem dodaje 1.

Akwizycja dynamiczna:
1. Najpierw robimy akwizycję statyczną pierwszego obrazu, następnie przesuwamy się w pamięci o obszar jednego obrazu i dokonujmy akwizycji statycznej, itd. Z góry zakładamy czas akwizycji każdego obrazu.

2. List mode pozwala na zapamiętanie wszystkich zdarzeń i uszeregowania ich w czasie. Zapamiętujemy X i Y które zaszły w określonym czasie (np. 10 ms).

Obrazy parametryczne - do określenia parametrów badanego obszaru

\end{document}