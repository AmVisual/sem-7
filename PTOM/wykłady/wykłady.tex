\documentclass{article}
\usepackage{polski}
\usepackage[margin=2.5cm]{geometry}

\setlength{\parindent}{0pt}

\title{Wykłady z Podstaw technik obrazowania w medycynie}
\author{Maciej Standerski}
\begin{document}
\maketitle

\section{Wykład (04.10.2023)}
\subsection{Przebieg i regulamin przedmiotu}

\textbf{Prowadzący}: dr inż Piotr Brzeski

\textbf{Konsultacje}: p. 60, 422 (IRiTM), poniedziałki, wtorki (rzadko), środy, czwartki (rzadko)

\textbf{e-mail}: piotr.brzeski@pw.edu.pl

\textbf{Laboratoria}: zaczynają sie pod koniec października

\textbf{Zaliczenie}:

Wykład:

Egzamin 0 i w sesji, 3 kolokwia (2h), średnia ocena z kolokwiów lub z egzaminu

Laboratoria:

1 laboratorium może być niezaliczone. W ocenę z laboratoriów wliczane są punkty ze wszystkich labów, również z niezaliczonych. Na ostatnich laboratoriach jest zaliczenie. Zajęcia laboratoryjne oceniane są na podstawie wejściówki oraz sprawozdania

Ocena końcowa $ = 0.5 O_W + 0.45 O_L + 0.05 O_{ZL}$, gdzie $O_W$ - ocena z wykładu, $O_L$ - ocena z laboratorium, $O_ZL$ - ocena z zaliczenia laboratorium

\textbf{Serwer}: Lkstudia3

\textbf{Literatura}: Wykłady z obrazowania medycznego na Politechnice Gdańskiej

\textbf{Harmonogram wykładu}:
\begin{itemize}
    \item 4 wykłady
    \item kolokwium 1
    \item 3 wykłady
    \item kolokwium 2
    \item 3 wykłady
    \item kolokwium 3
    \item egzamin
\end{itemize}


\subsection{Czym jest obrazowanie?}

Przedstawienie pewnej cechy fizycznej organizmu w postaci obrazu, zwykle 2D

(rozkład radiofarmaceutyku w tkankach – gamma kamera, scynytygrafia lub gammagrafia)

(gęstość protonów w tkankach – tomografia rezonansu magnetycznego)
Rzutowanie obiekt 3D (+ ewentualnie czas) na obraz 2D

\subsection{Cel obrazowania medycznego}
\begin{itemize}
    \item wgląd w anatomię struktur wewn organizmu i ich fizjologie
    \item analiza i interpretacja obrazów w diagnostyce
\end{itemize}

\subsection{Systemy obrazowania medycznego}

\begin{itemize}
    \item RTG,
    \item tomografia komputerowa,
    \item scyntygrafia, 
    \item tomografia PET, 
    \item tomografia rezonansu magnetycznego, 
    \item USG, 
    \item termografia,
    \item SPECT (tomografia pojedynczego fotonu)
\end{itemize}

\subsection{Od czego zależy wartość diagnostyczna?}

\begin{itemize}
    \item jakość obrazu (kontrast, rozdzielczość przestrzenna, stosunek sygnału użytecznego do szumu SNR, poziom artefaktów, poziom zniekształceń przestrzennych),
    \item warunki obserwacji,
    \item wiarygodność diagnostyczna,
    \item charakterystyka pracy lekarza-specjalisty
\end{itemize}

Rozdzielczość obrazowania - najmniejsza odległość w obiekcie obrazowanym między dwoma punktami o maksymalnym kontraście które można rozróżnić (FWHM – Full Width at Half Maximum, FWTM – Full Width at Tenth Maximum)

\end{document}