\documentclass{article}
\usepackage{polski}
\usepackage{hyperref}
\usepackage[margin=2.5cm]{geometry}

\setlength{\parindent}{0pt}

\title{Wykłady z Podstaw technik obrazowania w medycynie}
\author{Maciej Standerski}
\begin{document}
\maketitle

\section{Wykład (04.10.2023)}
\subsection{Przebieg i regulamin przedmiotu}

\textbf{Prowadzący}: dr inż Piotr Brzeski

\textbf{Konsultacje}: p. 60, 422 (IRiTM), poniedziałki, wtorki (rzadko), środy, czwartki (rzadko)

\textbf{Materiały online}: \href{https://studia.elka.pw.edu.pl/pl/23Z/-/login/}{https:\slash \slash studia.elka.pw.edu.pl\slash pl\slash 23Z\slash-\slash login\slash}

\textbf{e-mail}: piotr.brzeski@pw.edu.pl

\textbf{Laboratoria}: zaczynają sie pod koniec października

\textbf{Zaliczenie}:

Wykład:

Egzamin 0 i w sesji, 3 kolokwia (2h), średnia ocena z kolokwiów lub z egzaminu

Laboratoria:

1 laboratorium może być niezaliczone. W ocenę z laboratoriów wliczane są punkty ze wszystkich labów, również z niezaliczonych. Na ostatnich laboratoriach jest zaliczenie. Zajęcia laboratoryjne oceniane są na podstawie wejściówki oraz sprawozdania

Ocena końcowa $ = 0.5 O_W + 0.45 O_L + 0.05 O_{ZL}$, gdzie $O_W$ - ocena z wykładu, $O_L$ - ocena z laboratorium, $O_ZL$ - ocena z zaliczenia laboratorium

\textbf{Serwer}: Lkstudia3

\textbf{Literatura}: Wykłady z obrazowania medycznego na Politechnice Gdańskiej

\textbf{Harmonogram wykładu}:
\begin{itemize}
    \item 4 wykłady
    \item kolokwium 1
    \item 3 wykłady
    \item kolokwium 2
    \item 3 wykłady
    \item kolokwium 3
    \item egzamin
\end{itemize}


\subsection{Czym jest obrazowanie?}

Przedstawienie pewnej cechy fizycznej organizmu w postaci obrazu, zwykle 2D

(rozkład radiofarmaceutyku w tkankach – gamma kamera, scynytygrafia lub gammagrafia)

(gęstość protonów w tkankach – tomografia rezonansu magnetycznego)
Rzutowanie obiekt 3D (+ ewentualnie czas) na obraz 2D

\subsection{Cel obrazowania medycznego}
\begin{itemize}
    \item wgląd w anatomię struktur wewn organizmu i ich fizjologie
    \item analiza i interpretacja obrazów w diagnostyce
\end{itemize}

\subsection{Systemy obrazowania medycznego}

\begin{itemize}
    \item RTG,
    \item tomografia komputerowa,
    \item scyntygrafia, 
    \item tomografia PET, 
    \item tomografia rezonansu magnetycznego, 
    \item USG, 
    \item termografia,
    \item SPECT (tomografia pojedynczego fotonu)
\end{itemize}

\subsection{Od czego zależy wartość diagnostyczna?}

\begin{itemize}
    \item jakość obrazu (kontrast, rozdzielczość przestrzenna, stosunek sygnału użytecznego do szumu SNR, poziom artefaktów, poziom zniekształceń przestrzennych),
    \item warunki obserwacji,
    \item wiarygodność diagnostyczna,
    \item charakterystyka pracy lekarza-specjalisty
\end{itemize}

Rozdzielczość obrazowania - najmniejsza odległość w obiekcie obrazowanym między dwoma punktami o maksymalnym kontraście które można rozróżnić jako dwa obiekty na obrazie (FWHM – Full Width at Half Maximum, FWTM – Full Width at Tenth Maximum)

Rozdzielczość wartstwy może być większa (np. ok 1 mm)

Jakość obrazu - funkcja przenoszenia modulacji - iloraz między kontrastem fizycznym do kontrastu obrazowego $\mathrm{MTF} = \frac{K_f}{K_o}$ gdzie $K_f$ jest kontrastem fizycznego obiektu, natomiast $K_o$ to kontrast danego obiektu na obrazie.

Więcej o MTF i CTF (fukncji przenoszenia kontrastu) na stronie pod \href{https://brain.fuw.edu.pl/edu/index.php/Obrazowanie:Obrazowanie_Medyczne/Podstawowe_Parametry_Obraz%C3%B3w#label-fig:modulation_depth}{linkiem}

Wzorce w postaci sinusoidy.

$MTF = 1$ - gdy częstotliwość danego wzorca wynosi zero, obraz jest idealnie odwzorowany. Wraz z częstotliwością maleje funkcja przenoszenia modulacji.

Funkcja przenoszenia modulacji całkowitej jest iloczynem funkcji przenoszenia modulacji składowych systemu obrazowania.

$MTF_c = \displaystyle\prod_{i=1}^{N}\mathrm{MTF}_i$, gdzie $N$ - jest liczba systemów.

Na kolokwium

Stosunek sygnał/szum (NSR)

Artefakty

Zniekształcenia:
\begin{itemize}
    \item geometryczne
    \item zmiany pola magnetycznego
    \item przekrzywienie powierzchni odbiornika
    \item spowodowane fizjologią tkanek, narządów
\end{itemize}

\subsection{Warunki obserwacji}

Negatoskop - silna lampa do podświetlania negatywów z RTG

Wiarygodność diagnostyczna obrazów

Jeśli przesuniemy próg decyzyjny, może sprawić, że diagnoza będzie błędna.

Procedura decyzyjna

Macierz decyzyjna (prawdziwie pozytywna, fałszywie pozytywna, fałszywie negatywna, prawdziwie negatywna)

Wiarygodność metody diagnostyczna

Na kolokwium

$czulosc = TP/(TP+FN)$

$specyficznosc = TN/(TN+FP)$

Krzywa ROC - służy do oceny metody badawczej

Krzywa A - czysto przypadkowa

\section{Wykład (11.10.2023r.)}

\subsection{Techniki Medycyny Nuklearnej}

Funkcje:
\begin{itemize}
    \item terapia
    \item diagnostyka
\end{itemize}

Jak wprowadzić izotop promieniotwórczy do ciała:

- doustnie
- dożylnie
- przez układ oddechowy

Radiofarmaceutyk:
Podział w zależności od sposobu uzyskiwania:
- radionuklidy reaktorowe - powstające w wyniku reakcji jądrowej danego izotopu i neutronów
- radionuklidy cyklotronowe 
- radionuklidy generatorowe

Technet-99m (Tc-99m) generatorowy

Gammakamera

Jakie własności musi mieć dobry izotop promieniotwórczy
\begin{itemize}
    \item fizyczne \begin{itemize}
        \item $T_{1/2}$ - czas połowicznego rozpadu - musi być na tyle krótki, aby nie przekroczyć maksymalnej dawki dla pacjenta, oraz na tyle długi, aby móc go przetransporotwać. Dla przykładu czas połowicznego rozpadu dla technetu-99m wynosi 6h.
        \item $A [Bq]$ - jednostką w układzie SI jest Bekerel [$1 Bq = 1\mathrm{rozpad} / 1s$], dawniej jednostką aktywności były kiury [$1 Ci = 3,7 \cdot 10^{10} Bq$]. Aktywność promieniotwórcza, na tyle duża, aby promieniowanie mogło opóścić organizm pacjenta, ale też nie za duża, żeby nie przekroczyć dopuszczalnej dawki
        \item Rozpad gamma monoenergetyczny - najbardziej przenikliwy rodzaj promieniowania, dzięki temu możliwe jest obrazowanie na podstawie promieniowania wydostającego się z ciała pacjenta
        \item Energia - wykorzystywane są izotopy o energiach $30 - 360~keV$, najczęściej stosowany izotop - technet posiada dla przykładu energię $141~keV$
    \end{itemize}
    \item Chemiczne \begin{itemize}
        \item Czystość - ile izotopu promieniotwórczego zawiera dana próbka. Czysta izotop musi zawierać największą ilość tegoż izotopu, bez znaczącej ilości innych pierwiastków, czyli zanieczyszczeń. 
    \end{itemize}
    \item Biologiczne \begin{itemize}
        \item Powinowactwo do badanego organu\slash narządu - zdolność substancji do wiązania się z komórkami określonego narządu\slash tkanki\slash organu. Dzięki temu, że radiofarmaceutyki wiążą się z konkretnymi organami, gromadzą się one w nich i dzięki ich dużej koncentracji w określonym organie pozwala na analizę jego struktury i fizjologi. Różne radiofarmaceutyki mogą wiązać się z różnymi organami. Np. jod gromadzi się w tarczycy.
        \item Czas rozpadu biologicznego $T_B$ - jest to czas wydalania radioizotopu z organizmu. Pożądany czas rozpadu biologicznego musi być wystarczająco długi, aby móc wykonać badanie diagnostyczne, oraz na tyle krótki, aby pacjent szybko pozbył się farmaceutyku z organizmu po badaniu.
    \end{itemize}
\end{itemize}

Czas połowicznego rozpadu i czas rozpadu biologicznego razem składają się na tzw. efektywny czas połowicznego zaniku.   

\end{document}