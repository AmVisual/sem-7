

\documentclass{article}
\usepackage{polski}

\usepackage{hyperref}
\usepackage[margin=2.5cm]{geometry}

\setlength{\parindent}{0pt}

\title{Wstęp do Fizyki Medycznej - wykłady}
\author{Maciej Standerski}
\begin{document}
\maketitle

\section{Wykład (08.11.2023)}

Andrzej Hrynkiewicz "Człowiek i promieniowanie jaonizujące",

Ustawa Prawo Atomowe

Natalia Golnik "Radiologia",

Janusz Henschke "Ochrona Radiologiczna",

Tadeusz Musiałowicz, "Słownik Terminów Ochrony Radiologicznej" CLOR

Zaliczenie:

Wykłady: kolokwium na ostatnich zajęciach (waga 60\%)

Laboratorium: Ocena końcowa jest średnią z ocen cząstkowych za każde z ćwiczeń. Obecność bez oddania sprawozdania to ocena 2. Nieusprawiedliwona nieobecność oznacza ocenę 0 za dane ćw.

Min en. promieniowania jonizującego dla atomu wodoru: 13,6 eV

Promieniowanie jonizujące - promieniowanie składające się z cząstek bezpośrednio lub porśrednio jonizujących albo z obu rodzajów tych cząstek lub fal elektromagnetycznych o długości do 100 nm. (min en. ok. 124.2 eV).

Źródła promieniowania jonizującego:
Naturalne: skorupa ziemska, promieniowanie kosmiczne, żywność i organizmy żywe

Sztuczne: aparaty rentgenowskie, akceleratory medyczne i przemysłowe, raektory jądrowe, wytworzone sztucznie źródła promieniotwórcze, w tym radiofarmaceutyki

Terapia śródoperacyjna - naświetla się mmiejsce po wycięciu guza by pozbyć się ewentualnych pozostałych komórek nowotworowych. Akcelerator śródoperacyjny intraLine

RTG + aktywacja materiałów nautronami oraz aktywacji promieniowaniem X o wysokiej energii.

\subsection{Dozymetria i ochrona radiologiczna}

Dozymetria - dział fizyki zajmujący się pomiraem i obliczaniem dawek promieniowania jonizującego oraz ocenę skażeń promieniotwórczych.

Ochrona radiologiczna - zapobieganie narażeniu ludzi i skażeniu środowiska, ograniczenie ich skutkoów do poziomu tak niskiegom, jak tylko jest to rozsądnie osiągalne, przy uwzględnieniu czynników ekonomicznych, społecznych i zdrowotnych.

Ochrona raiologiczna pacjenta - zespół czynności i ograniczeń zmierzających do zminimalizowania narażenia pacjenta na promieniowanie jonizujące.

\subsection{Zalecenia i standardy}

International Commision on Radiological Protection - ICRP

International Commision on Radiation Units \& Measurements - ICRU

Obowiązujące akty prawne i zalecenia: Ustawa Prawo Atomowe, Rozporządzenie Rady Ministrów, Dyrektywa EURATOM, Recommendations of the International Commision on Radiological Protection

\subsection{Podstawowe wielkości dozymetryczne}

\begin{itemize}
    \item Wielkości fizyczne: Aktywność źródła, fluencja cząstek, gęstość strumienia cząstek
    \item Wielkości dozymetryczne: kerma, dawka pochłonięta, moc dawki pochłoniętej, energia przekazana
\end{itemize}

Dawka to ogólnie miara energii zdeponowanej przez promieniowanie w materiale tarczowym.

W zależności od kontekstu dawka pochłonięta, równoważna dawka obciążająca, efektywna dawka obciążająca, dawka efektywna, dawka równoważna lub dawka w narządzie.

Aktywność źródła promieniotwórczego: $A = \Delta N / \Delta t$, aktywność promieniotwórcza zmienia się w czasie. [$Bq/kg, Bq/m^2, Bq/m^3$]

$\Delta N = - \lambda N \Delta t$

Fluencja cząstek - liczba padających cząsten dN na powierchnię da, gdzie dNjest liczbą cząstek, które przechodzą przez powierchnię sferyczną da.

Gęstość strumienia cząstek

Kerma - suma energii kinetycznej wszystkich cząstek naładowanych uwolnionych przez cząstki nienaładowane w jednostce masy ośrodka [$Gy = J/kg$]

Dawka pochłonięta - średnia energia przekazana prze promieniowanie w jednostce masy danego ciała $D = \frac{d\epsilon}{dm}$

Moc dawki pochłoniętej: $\dot{D} = \frac{dD}{dt}$

Dawka pochłonięta vs Kerma

Do kermy wliczamy całkowitą energię cząstek naładowanych wygenerowanych w masie dm

Samej dawki nie da się zmierzyć. Mierzymy kermę, i następnie na podstawie współczynników przeliczamy ją na dawkę.

Ekspozycja (dawka ekspozycyjna.) $X = dQ / dm$

Energia przekazana $\epsilon = \sum R_{in} - \sum R_{out} + \sum Q$

\subsection{Rodzaje promieniowania}

Wysokoenergetyczne promieniowanie E-M (X i $\gamma$)

Promieniowanie $\beta$

Cząstki alfa

Protony i ciężkie jony, w tym fragmenty jądrowe

Neutrony

Boron neutron capture therapy - podobne działanie jak przy badaniu z wykorzystaniem radiofarmaceutyku. Zamiast dostarczania radiofarmaceutyku do nowotworu, umieszczamy w tkance boron, i następnie naświetlamy neutronami. Powstaje cząstka w miejscu lokalizacji boronu. Cząstki alfa niszczą tkankę nowotworową.

Dawka równoważna - dawka pochłonięta w tkance lub narżadzie T, ważona dla rodzaju i energii promieniowania R: $H_{T,R} = w_RD_{T,R} [Sv]$, $H_{T} = \sum_{R} w_RD_{T,R} [Sv]$

$w_r$ największe dla cząstek alfa (daje większey efekt biologiczny)

Fotony - 1
El. i miony 1
Protony i piony - 2
Cząstki alfa, fragmenty rozszczepienia, ciężkie jony - 20

Spektrum energetyczne (współczynnik wagowy dla neutronów) (neutrony o energiach w ok. 1 MeV mają największe szanse aby doszło do reakcji jądrowej, przez co mogą spowodować więskze szkody. Dla tych neutronów współczynnik wagowy jest największy. Natomiast neutrony o energiach mniejszych lub większych mają mniejszą szansę na wywołanie reakcji, przez co współczynniki dla tych neutronów są mniejsze.)

Dawka skuteczna: $E = \sum_T w_T \sum w_R D_{T,R}$, tkanki mają różną promieniowrażliwość. Komórki, które się kształtują, rosną mają większą promieniowrażliwość niż komórekrki rozwinięte. Dlatego radioterapia działa, ponieważ komórki nowotworowe szybko rosną i są bardziej podatne na działanie promieniowania. Jest to również powód, dlaczego nie należy przeprowadzać radioterapii kobietom w ciąży, ponieważ promieniowanie może uszkodzić rozwijający się płód.

Im większy współczynnik $w_T$, tym większe skutki sktochastyczne.

Wartości czynnika wagowego

Kula LCRU - fantom, składa się z tlenu (76,2\%), węgla (11,1\%), wodoru (10,1\%) i azotu (2,6\%).

Wartości czynnika wagowego: $Q = \frac{1}{D}\int_{L=0}^{\infty} Q(L)D_LdL$, D - dawka pochłonięta, L - nieograniczone liniowe przekazanie energii (LET) na jeden mikrometr toru cząstki naładowanej w wodzie.

Dawka skuteczna E otrzymana w ciągu określonego czasu

$E = E_z + \sum_j e(g)_{j,p}J_{j,p} + \sum_j e(g)_{j,o}J_{j,o}$

e - obciążająca dawka skuteczna (efektywna) - zależą od sposobu przechodzenia izotopu do przewodu pokarmowego (p) i z, do i z drogi oddechowej (o).
J - aktywność izotopu

Dawka pochłonięta w narządzie T -$>$ współczynnik wagowy promieniowania -$>$ dawka równoważna w narządzie T -$>$ współczynnik wagowy tkanki -$>$ dawka skuteczna (efektywna)

Dawka pochłonięta -$>$ współczynnik jakości promieniowania -$>$ równoważnik dawki -$>$ wielkości robocze (przestrzenny równoważnik dawki $H^*(d)$, kierunkowy równoważnik dawki $H'(d,\Omega)$, indywidualny rownoważnik dawki)

Dozymetria awaryjna - sprawdzanie ilości mutacji w komórkach (zmieny chromosomalne - dicentyki, czyli dwa miejsca połączenia - błędnie naprawione chromosomy. Zliczając ilość takich dicentryków można oszacować dawkę promieniowania, potrzbne są wowczas bazy danych, które określają liczbę dicentryków w zależności od dawki)

system wielkości dozymetrycznych roboczych

\section{Wykład 15.11.2023}

11 stycznia - laboratorium z RTG

Cząstki naładowane łatwiej jest zidentyfikować.

\subsection{Oddziaływanie promieniowania z materią}

Prawo osłabienia wiązki promieniowania $n = n_0 e^{-\mu_x x}$, $\mu_x = \sigma * N$ - liniowy współczynnik osłabienia wiązki, $\mu_d = \mu_x / \rho$ - masowy współczynik osłabienia wiązki (dla materiałów o niejednorodnej gęstości ). $\rho$ - masowy współczynnik osłabienia wiązki.

Oddziaływanie ciężkie naładowane (wzór Bethego-Blocha na slajdach z wykładu), cząstki tracą energie głównie z powodu oddziaływania kulombowskiego

Krzywa jonizacji Bragga - jonizacja w ośrodkju w funkcji przebytej drogi

Krzywa zasięgu w materiale

Zasięg i jonizacja w materiałach

- dla cząstek alfa zasięgi rzędu cm

Straty energii elektronów - rozproszenie na el. zmienia się kierunek lotu elektronu, ale zmienia energię elektronu

Rozpraszanie na elektronach $Z/\beta^4$ i na jądrach atomowych $Z^2/\beta^4$.

Promieniowanie hamowania (Brehmstrallung) ($E_\gamma \propto E/m_0c^2$) - wynika z oddziaływania el. z polem elektrycznym jąder atomowych

Oddziaływanie fotonów w materii (zjawisko fotoelektryczne, efekt Comptona, tworzenie par elektron-pozyton)

Zasięg kwantów gamma

Kwanty gamma są promieniowaniem jonizującym pośrednio (tylko 10\% oddziaływania bezposrednio, reszta energii jest przekazywana przez pośrednie zjawiska, reakcje)

Współczynnik osłabienia: $\mu_0 = \mu_{photo} + \mu_{compt} + \mu_{pair}$

$\mu_0 = \mu_{photo}f_{photo} + \mu_{compt}f_{compt} + \mu_{pair}f_{pair}$

Neutrony - procesy jonizacji zachodzą w wyniku wtórnych procesów

Przekrój czynny silnie zależy od energii neutronu i rodzaju materiału.

Sposób na efektywny pomiar dawki jest spowolnienie neutronów a później ich detekcja w wyniku reakcji jądrowych.

\subsection{Metody detektcji neutronów}

Detektor Albedo (poszukać informacji)

Absorpcja neutronów: $I = I_0 e^{-N\sigma d}$, N - koncentracja jąder na jedn objętości, $\sigma$ - przekrój czynny, d - grubość absorbenta

Detekcja neutronów termicznych - spowolnić neutorny do energii rzędu elektornowoltów, a następnie rejestrujemy rekacje jądrowe

Detektor TLD

Hel-4 jako scyntylator

\subsection{Detektory wykorzystywane w dozymetrii}

Typy detektorów

Zależy od rodzaju promieniowania, jakie chcemy mierzyć, źródła promieniowania

Detektory aktywne - pomiar w czasie rzeczywistym, wymagają zasilania

Detektory pasywne - całkują informację w czasie, wymagają układu do odczytu, nie wymagają zasilania

Cechy detektorów: wydajność, czas martwy, zdolność rozdzielcza (czasowa, przestrzenna, energetyczna), efekty elektroniki (szumy, wzmocnienie), czy wymagają kalibracji i odjęcia tła ("bieg własny").

Detektor składa się z: detektora właściwego, czyli substancji czułej na promieniowanie; systemu prztwarzania sygnału, części pozadetekcyjne (obudowa, mocowanie)

Wydajność - stosunek ilości cząstek, które mogę zarejestrować do liczby cząstek emitowanych przez źródło.

Czas martwy detektora - czas po detekcji gdy detektor jest nieczuły ($D = 1 - \frac{N}{T}\tau$)

Wyznaczanie wydajności - porównanie z układem, w którym znamy liczbę cząstek emitowanych (intensywnością źródła)

Druga metoda pomiaru - koincydencja dla dwch detektorów o nieznanej wydajności

Zdolność rozdzielcza - wykorzystujemy spektrum energii wytwarzane przez pierwiastki promieniotwórcze, najważniejsza w dozymetrii jest rozdzielczość energetyczna (niepewność liczby zliczeń maleje wraz ze wzrostem liczby zliczeń - zgodnie z rozkładem Poissona). Zdolności rozdzielcze: dla ciał stałych 3 eV, gazowych 30 eV, dla scyntylatorów 300 eV

Kalibracja\slash wzorcowanie

Wzorcowanie przyrządów - Laboratorium Wzorcowania Przyrządów Dozymetrycznych i Radonowych (LWPDiR) CLOR

Detektory gazowe: potrzebne jest minimalne napięcie, które rozpędzi elektrony, jeśli będzie ono za mało, może dojść do rekombinacji elektronów, i zliczenie nie zostanie zarejestrowane

Zakres roboczy napięcia: obszar rekombinacji - komora jonizacyjna - licznik proporcjonalny - obszar ograniczonej proporcjonalności - licznik Geigera-Mulera - obszar wyładowania ciągłego

Instytucje nadzorujące skażenia promieniotwórcze: Państwowa Agencja Atomistyki, IMGW, MON - Ministerstwo Obrony Narodowej



\end{document}
